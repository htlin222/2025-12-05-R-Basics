% Options for packages loaded elsewhere
% Options for packages loaded elsewhere
\PassOptionsToPackage{unicode}{hyperref}
\PassOptionsToPackage{hyphens}{url}
\PassOptionsToPackage{dvipsnames,svgnames,x11names}{xcolor}
%
\documentclass[
  chinese,
  letterpaper,
  DIV=11,
  numbers=noendperiod]{scrreprt}
\usepackage{xcolor}
\usepackage{amsmath,amssymb}
\setcounter{secnumdepth}{5}
\usepackage{iftex}
\ifPDFTeX
  \usepackage[T1]{fontenc}
  \usepackage[utf8]{inputenc}
  \usepackage{textcomp} % provide euro and other symbols
\else % if luatex or xetex
  \usepackage{unicode-math} % this also loads fontspec
  \defaultfontfeatures{Scale=MatchLowercase}
  \defaultfontfeatures[\rmfamily]{Ligatures=TeX,Scale=1}
\fi
\usepackage{lmodern}
\ifPDFTeX\else
  % xetex/luatex font selection
\fi
% Use upquote if available, for straight quotes in verbatim environments
\IfFileExists{upquote.sty}{\usepackage{upquote}}{}
\IfFileExists{microtype.sty}{% use microtype if available
  \usepackage[]{microtype}
  \UseMicrotypeSet[protrusion]{basicmath} % disable protrusion for tt fonts
}{}
\makeatletter
\@ifundefined{KOMAClassName}{% if non-KOMA class
  \IfFileExists{parskip.sty}{%
    \usepackage{parskip}
  }{% else
    \setlength{\parindent}{0pt}
    \setlength{\parskip}{6pt plus 2pt minus 1pt}}
}{% if KOMA class
  \KOMAoptions{parskip=half}}
\makeatother
% Make \paragraph and \subparagraph free-standing
\makeatletter
\ifx\paragraph\undefined\else
  \let\oldparagraph\paragraph
  \renewcommand{\paragraph}{
    \@ifstar
      \xxxParagraphStar
      \xxxParagraphNoStar
  }
  \newcommand{\xxxParagraphStar}[1]{\oldparagraph*{#1}\mbox{}}
  \newcommand{\xxxParagraphNoStar}[1]{\oldparagraph{#1}\mbox{}}
\fi
\ifx\subparagraph\undefined\else
  \let\oldsubparagraph\subparagraph
  \renewcommand{\subparagraph}{
    \@ifstar
      \xxxSubParagraphStar
      \xxxSubParagraphNoStar
  }
  \newcommand{\xxxSubParagraphStar}[1]{\oldsubparagraph*{#1}\mbox{}}
  \newcommand{\xxxSubParagraphNoStar}[1]{\oldsubparagraph{#1}\mbox{}}
\fi
\makeatother

\usepackage{color}
\usepackage{fancyvrb}
\newcommand{\VerbBar}{|}
\newcommand{\VERB}{\Verb[commandchars=\\\{\}]}
\DefineVerbatimEnvironment{Highlighting}{Verbatim}{commandchars=\\\{\}}
% Add ',fontsize=\small' for more characters per line
\usepackage{framed}
\definecolor{shadecolor}{RGB}{241,243,245}
\newenvironment{Shaded}{\begin{snugshade}}{\end{snugshade}}
\newcommand{\AlertTok}[1]{\textcolor[rgb]{0.68,0.00,0.00}{#1}}
\newcommand{\AnnotationTok}[1]{\textcolor[rgb]{0.37,0.37,0.37}{#1}}
\newcommand{\AttributeTok}[1]{\textcolor[rgb]{0.40,0.45,0.13}{#1}}
\newcommand{\BaseNTok}[1]{\textcolor[rgb]{0.68,0.00,0.00}{#1}}
\newcommand{\BuiltInTok}[1]{\textcolor[rgb]{0.00,0.23,0.31}{#1}}
\newcommand{\CharTok}[1]{\textcolor[rgb]{0.13,0.47,0.30}{#1}}
\newcommand{\CommentTok}[1]{\textcolor[rgb]{0.37,0.37,0.37}{#1}}
\newcommand{\CommentVarTok}[1]{\textcolor[rgb]{0.37,0.37,0.37}{\textit{#1}}}
\newcommand{\ConstantTok}[1]{\textcolor[rgb]{0.56,0.35,0.01}{#1}}
\newcommand{\ControlFlowTok}[1]{\textcolor[rgb]{0.00,0.23,0.31}{\textbf{#1}}}
\newcommand{\DataTypeTok}[1]{\textcolor[rgb]{0.68,0.00,0.00}{#1}}
\newcommand{\DecValTok}[1]{\textcolor[rgb]{0.68,0.00,0.00}{#1}}
\newcommand{\DocumentationTok}[1]{\textcolor[rgb]{0.37,0.37,0.37}{\textit{#1}}}
\newcommand{\ErrorTok}[1]{\textcolor[rgb]{0.68,0.00,0.00}{#1}}
\newcommand{\ExtensionTok}[1]{\textcolor[rgb]{0.00,0.23,0.31}{#1}}
\newcommand{\FloatTok}[1]{\textcolor[rgb]{0.68,0.00,0.00}{#1}}
\newcommand{\FunctionTok}[1]{\textcolor[rgb]{0.28,0.35,0.67}{#1}}
\newcommand{\ImportTok}[1]{\textcolor[rgb]{0.00,0.46,0.62}{#1}}
\newcommand{\InformationTok}[1]{\textcolor[rgb]{0.37,0.37,0.37}{#1}}
\newcommand{\KeywordTok}[1]{\textcolor[rgb]{0.00,0.23,0.31}{\textbf{#1}}}
\newcommand{\NormalTok}[1]{\textcolor[rgb]{0.00,0.23,0.31}{#1}}
\newcommand{\OperatorTok}[1]{\textcolor[rgb]{0.37,0.37,0.37}{#1}}
\newcommand{\OtherTok}[1]{\textcolor[rgb]{0.00,0.23,0.31}{#1}}
\newcommand{\PreprocessorTok}[1]{\textcolor[rgb]{0.68,0.00,0.00}{#1}}
\newcommand{\RegionMarkerTok}[1]{\textcolor[rgb]{0.00,0.23,0.31}{#1}}
\newcommand{\SpecialCharTok}[1]{\textcolor[rgb]{0.37,0.37,0.37}{#1}}
\newcommand{\SpecialStringTok}[1]{\textcolor[rgb]{0.13,0.47,0.30}{#1}}
\newcommand{\StringTok}[1]{\textcolor[rgb]{0.13,0.47,0.30}{#1}}
\newcommand{\VariableTok}[1]{\textcolor[rgb]{0.07,0.07,0.07}{#1}}
\newcommand{\VerbatimStringTok}[1]{\textcolor[rgb]{0.13,0.47,0.30}{#1}}
\newcommand{\WarningTok}[1]{\textcolor[rgb]{0.37,0.37,0.37}{\textit{#1}}}

\usepackage{longtable,booktabs,array}
\usepackage{calc} % for calculating minipage widths
% Correct order of tables after \paragraph or \subparagraph
\usepackage{etoolbox}
\makeatletter
\patchcmd\longtable{\par}{\if@noskipsec\mbox{}\fi\par}{}{}
\makeatother
% Allow footnotes in longtable head/foot
\IfFileExists{footnotehyper.sty}{\usepackage{footnotehyper}}{\usepackage{footnote}}
\makesavenoteenv{longtable}
\usepackage{graphicx}
\makeatletter
\newsavebox\pandoc@box
\newcommand*\pandocbounded[1]{% scales image to fit in text height/width
  \sbox\pandoc@box{#1}%
  \Gscale@div\@tempa{\textheight}{\dimexpr\ht\pandoc@box+\dp\pandoc@box\relax}%
  \Gscale@div\@tempb{\linewidth}{\wd\pandoc@box}%
  \ifdim\@tempb\p@<\@tempa\p@\let\@tempa\@tempb\fi% select the smaller of both
  \ifdim\@tempa\p@<\p@\scalebox{\@tempa}{\usebox\pandoc@box}%
  \else\usebox{\pandoc@box}%
  \fi%
}
% Set default figure placement to htbp
\def\fps@figure{htbp}
\makeatother



\ifLuaTeX
\usepackage[bidi=basic,provide=*]{babel}
\else
\usepackage[bidi=default,provide=*]{babel}
\fi
% get rid of language-specific shorthands (see #6817):
\let\LanguageShortHands\languageshorthands
\def\languageshorthands#1{}


\setlength{\emergencystretch}{3em} % prevent overfull lines

\providecommand{\tightlist}{%
  \setlength{\itemsep}{0pt}\setlength{\parskip}{0pt}}



 


\usepackage{booktabs}
\usepackage{caption}
\usepackage{longtable}
\usepackage{colortbl}
\usepackage{array}
\usepackage{anyfontsize}
\usepackage{multirow}
\KOMAoption{captions}{tableheading}
\makeatletter
\@ifpackageloaded{bookmark}{}{\usepackage{bookmark}}
\makeatother
\makeatletter
\@ifpackageloaded{caption}{}{\usepackage{caption}}
\AtBeginDocument{%
\ifdefined\contentsname
  \renewcommand*\contentsname{目錄}
\else
  \newcommand\contentsname{目錄}
\fi
\ifdefined\listfigurename
  \renewcommand*\listfigurename{圖目錄}
\else
  \newcommand\listfigurename{圖目錄}
\fi
\ifdefined\listtablename
  \renewcommand*\listtablename{表目錄}
\else
  \newcommand\listtablename{表目錄}
\fi
\ifdefined\figurename
  \renewcommand*\figurename{圖}
\else
  \newcommand\figurename{圖}
\fi
\ifdefined\tablename
  \renewcommand*\tablename{表}
\else
  \newcommand\tablename{表}
\fi
}
\@ifpackageloaded{float}{}{\usepackage{float}}
\floatstyle{ruled}
\@ifundefined{c@chapter}{\newfloat{codelisting}{h}{lop}}{\newfloat{codelisting}{h}{lop}[chapter]}
\floatname{codelisting}{列表}
\newcommand*\listoflistings{\listof{codelisting}{列表目錄}}
\makeatother
\makeatletter
\makeatother
\makeatletter
\@ifpackageloaded{caption}{}{\usepackage{caption}}
\@ifpackageloaded{subcaption}{}{\usepackage{subcaption}}
\makeatother
\usepackage{bookmark}
\IfFileExists{xurl.sty}{\usepackage{xurl}}{} % add URL line breaks if available
\urlstyle{same}
\hypersetup{
  pdftitle={R 語言統計工具入門:用 AI 學會做臨床研究統計},
  pdfauthor={林協霆},
  pdflang={zh-TW},
  colorlinks=true,
  linkcolor={blue},
  filecolor={Maroon},
  citecolor={Blue},
  urlcolor={Blue},
  pdfcreator={LaTeX via pandoc}}


\title{R 語言統計工具入門:用 AI 學會做臨床研究統計}
\author{林協霆}
\date{2025-12-05}
\begin{document}
\maketitle

\renewcommand*\contentsname{目錄}
{
\hypersetup{linkcolor=}
\setcounter{tocdepth}{2}
\tableofcontents
}

\bookmarksetup{startatroot}

\chapter*{前言}\label{ux524dux8a00}
\addcontentsline{toc}{chapter}{前言}

\markboth{前言}{前言}

歡迎來到「R 語言統計工具入門:用 AI 學會做臨床研究統計」!

\begin{itemize}
\tightlist
\item
  \href{https://htlin222.github.io/learn-r-with-ai/}{本書連結}
\item
  \href{https://github.com/htlin222/learn-r-with-ai}{Github Repo} 以及
  \href{https://github.com/htlin222/learn-r-with-ai/archive/refs/heads/main.zip}{下載程式碼的連結}
\end{itemize}

\section*{課程目標}\label{ux8ab2ux7a0bux76eeux6a19}
\addcontentsline{toc}{section}{課程目標}

\markright{課程目標}

今天結束時,你將能夠:

\begin{itemize}
\tightlist
\item
  ✅ 把問題描述清楚,讓 AI 幫你寫程式
\item
  ✅ 看懂 AI 給的程式碼大概在做什麼
\item
  ✅ 當程式出錯,知道怎麼問 AI 修正
\item
  ✅ 產出可以放進論文的表格和圖表
\item
  ✅ 有一個可以重複使用的分析範本
\end{itemize}

\section*{這堂課的玩法}\label{ux9019ux5802ux8ab2ux7684ux73a9ux6cd5}
\addcontentsline{toc}{section}{這堂課的玩法}

\markright{這堂課的玩法}

\begin{enumerate}
\def\labelenumi{\arabic{enumi}.}
\tightlist
\item
  我給你一個「任務」
\item
  你把任務描述貼給 AI(ChatGPT / Claude)
\item
  AI 給你程式碼
\item
  你貼到 RStudio / Antigravity 執行
\item
  我們一起看結果、理解發生了什麼
\end{enumerate}

\textbf{記住:你的工作是「問對問題」,不是「寫對程式」。}

\section*{資料說明}\label{ux8cc7ux6599ux8aaaux660e}
\addcontentsline{toc}{section}{資料說明}

\markright{資料說明}

本課程使用 \texttt{patient\_data.csv} 檔案,包含以下欄位:

\begin{itemize}
\tightlist
\item
  \texttt{patient\_id}:病人編號
\item
  \texttt{treatment}:治療組別(A 或 B)
\item
  \texttt{age}:年齡
\item
  \texttt{gender}:性別(M 或 F)
\item
  \texttt{los}:住院天數(length of stay)
\end{itemize}

這份資料將貫穿整個課程,讓你能夠實際操作並產出真實可用的統計分析結果。

\bookmarksetup{startatroot}

\chapter{第一部分:你今天就會寫程式}\label{ux7b2cux4e00ux90e8ux5206ux4f60ux4ecaux5929ux5c31ux6703ux5bebux7a0bux5f0f}

本部分預計時間:40 分鐘

\section{任務
1:你的第一個程式}\label{ux4efbux52d9-1ux4f60ux7684ux7b2cux4e00ux500bux7a0bux5f0f}

📋 \textbf{複製這段話,貼給 AI:}

\begin{quote}
我完全沒學過 R 語言。請給我一段最簡單的 R 程式碼,讓我可以畫出一個有 5
根柱子的長條圖,每根柱子的高度分別是
10、25、15、30、20。請給我可以直接複製貼上執行的程式碼。
\end{quote}

\subsection{範例程式碼}\label{ux7bc4ux4f8bux7a0bux5f0fux78bc}

\begin{Shaded}
\begin{Highlighting}[]
\CommentTok{\# 建立資料}
\NormalTok{heights }\OtherTok{\textless{}{-}} \FunctionTok{c}\NormalTok{(}\DecValTok{10}\NormalTok{, }\DecValTok{25}\NormalTok{, }\DecValTok{15}\NormalTok{, }\DecValTok{30}\NormalTok{, }\DecValTok{20}\NormalTok{)}
\NormalTok{categories }\OtherTok{\textless{}{-}} \FunctionTok{c}\NormalTok{(}\StringTok{"A"}\NormalTok{, }\StringTok{"B"}\NormalTok{, }\StringTok{"C"}\NormalTok{, }\StringTok{"D"}\NormalTok{, }\StringTok{"E"}\NormalTok{)}

\CommentTok{\# 畫長條圖}
\FunctionTok{barplot}\NormalTok{(heights, }
        \AttributeTok{names.arg =}\NormalTok{ categories,}
        \AttributeTok{main =} \StringTok{"我的第一個 R 圖表"}\NormalTok{,}
        \AttributeTok{xlab =} \StringTok{"類別"}\NormalTok{,}
        \AttributeTok{ylab =} \StringTok{"高度"}\NormalTok{,}
        \AttributeTok{col =} \StringTok{"steelblue"}\NormalTok{)}
\end{Highlighting}
\end{Shaded}

\pandocbounded{\includegraphics[keepaspectratio]{part1_files/figure-pdf/first-plot-1.pdf}}

⏱️ 3 分鐘後,你應該會在 RStudio / Antigravity 看到一張圖。

\section{任務
2:發生了什麼事?}\label{ux4efbux52d9-2ux767cux751fux4e86ux4ec0ux9ebcux4e8b}

📋 \textbf{複製這段話,貼給 AI:}

\begin{quote}
請用「教小學生」的方式,解釋你剛剛給我的那段 R
程式碼。每一行在做什麼?什麼是「變數」?什麼是「函數」?
\end{quote}

\subsection{程式碼解釋}\label{ux7a0bux5f0fux78bcux89e3ux91cb}

\begin{itemize}
\tightlist
\item
  \textbf{變數}:就像一個盒子,可以裝東西。\texttt{heights} 裝了 5
  個數字。
\item
  \textbf{函數}:就像一個機器,你給它材料,它產出結果。\texttt{barplot()}
  就是畫圖的機器。
\item
  \textbf{參數}:就像機器的設定,告訴機器怎麼運作。
\end{itemize}

\section{任務
3:當程式出錯時}\label{ux4efbux52d9-3ux7576ux7a0bux5f0fux51faux932fux6642}

📋 \textbf{如果剛剛有錯誤訊息,複製錯誤訊息,加上這段話貼給 AI:}

\begin{quote}
我在 RStudio / Antigravity 執行你給的程式碼,出現這個錯誤:
【貼上錯誤訊息】 請告訴我這是什麼意思,以及怎麼修正。
\end{quote}

\subsection{常見錯誤範例}\label{ux5e38ux898bux932fux8aa4ux7bc4ux4f8b}

\begin{Shaded}
\begin{Highlighting}[]
\CommentTok{\# 這會產生錯誤(故意的)}
\FunctionTok{barplot}\NormalTok{(height)  }\CommentTok{\# 變數名稱打錯了}
\end{Highlighting}
\end{Shaded}

🎯 記住:錯誤訊息是線索,不是你的錯。

\section{任務 4:認識套件}\label{ux4efbux52d9-4ux8a8dux8b58ux5957ux4ef6}

📋 \textbf{複製這段話,貼給 AI:}

\begin{quote}
R 語言裡面的「套件」(package) 是什麼?可以用「手機
App」來比喻嗎?另外,\texttt{library()} 這個指令是在做什麼?
\end{quote}

\subsection{套件說明}\label{ux5957ux4ef6ux8aaaux660e}

\begin{itemize}
\tightlist
\item
  \textbf{套件 = 手機 App}:提供額外功能
\item
  \textbf{安裝套件 = 下載 App}:\texttt{install.packages("套件名稱")}
\item
  \textbf{載入套件 = 開啟 App}:\texttt{library(套件名稱)}
\end{itemize}

\section{任務
5:安裝我們需要的工具}\label{ux4efbux52d9-5ux5b89ux88ddux6211ux5011ux9700ux8981ux7684ux5de5ux5177}

📋 \textbf{複製這段話,貼給 AI:}

\begin{quote}
我要在 R 裡面使用 ggplot2、gtsummary、dplyr
這三個套件。請給我安裝這些套件的程式碼,以及安裝完之後怎麼載入它們。
\end{quote}

\subsection{安裝與載入套件}\label{ux5b89ux88ddux8207ux8f09ux5165ux5957ux4ef6}

\begin{Shaded}
\begin{Highlighting}[]
\CommentTok{\# 安裝套件(只需執行一次)}
\FunctionTok{install.packages}\NormalTok{(}\FunctionTok{c}\NormalTok{(}\StringTok{"ggplot2"}\NormalTok{, }\StringTok{"gtsummary"}\NormalTok{, }\StringTok{"dplyr"}\NormalTok{))}

\CommentTok{\# 載入套件(每次使用前都要執行)}
\FunctionTok{library}\NormalTok{(ggplot2)}
\FunctionTok{library}\NormalTok{(gtsummary)}
\FunctionTok{library}\NormalTok{(dplyr)}
\end{Highlighting}
\end{Shaded}

⏱️ 安裝可能需要 2-3 分鐘,這是正常的。

\bookmarksetup{startatroot}

\chapter{第二部分:讀取你的資料}\label{ux7b2cux4e8cux90e8ux5206ux8b80ux53d6ux4f60ux7684ux8cc7ux6599}

本部分預計時間:20 分鐘

\section{任務 6:讀取 CSV
檔案}\label{ux4efbux52d9-6ux8b80ux53d6-csv-ux6a94ux6848}

📋 \textbf{複製這段話,貼給 AI:}

\begin{quote}
我有一個 CSV 檔案叫做 patient\_data.csv,放在我的桌面。我正在用 RStudio
/ Antigravity。請告訴我:

\begin{enumerate}
\def\labelenumi{\arabic{enumi}.}
\tightlist
\item
  怎麼讀取這個檔案到 R 裡面,並存成一個叫做 \texttt{my\_data} 的變數
\item
  怎麼看前幾筆資料確認有讀成功
\end{enumerate}

請給我可以直接執行的程式碼。
\end{quote}

\subsection{讀取資料}\label{ux8b80ux53d6ux8cc7ux6599}

\begin{Shaded}
\begin{Highlighting}[]
\CommentTok{\# 讀取 CSV 檔案}
\NormalTok{my\_data }\OtherTok{\textless{}{-}} \FunctionTok{read.csv}\NormalTok{(}\StringTok{"patient\_data.csv"}\NormalTok{)}

\CommentTok{\# 顯示前 6 筆資料}
\FunctionTok{head}\NormalTok{(my\_data)}
\end{Highlighting}
\end{Shaded}

\begin{verbatim}
  patient_id treatment age gender los
1          1         A  45      M   5
2          2         B  52      F   8
3          3         A  38      M   4
4          4         B  61      F  12
5          5         A  55      F   6
6          6         B  42      M   9
\end{verbatim}

\section{任務
7:認識你的資料}\label{ux4efbux52d9-7ux8a8dux8b58ux4f60ux7684ux8cc7ux6599}

📋 \textbf{複製這段話,貼給 AI:}

\begin{quote}
我已經把資料讀進來了,存在一個叫做 \texttt{my\_data}
的變數。請給我程式碼,讓我可以:

\begin{enumerate}
\def\labelenumi{\arabic{enumi}.}
\tightlist
\item
  看這個資料有幾列幾欄
\item
  看每個欄位的基本統計(平均、最大、最小等)
\item
  看每個欄位是什麼資料類型(數字、文字等)
\end{enumerate}
\end{quote}

\subsection{探索資料}\label{ux63a2ux7d22ux8cc7ux6599}

\begin{Shaded}
\begin{Highlighting}[]
\CommentTok{\# 查看資料維度(幾列幾欄)}
\FunctionTok{dim}\NormalTok{(my\_data)}
\end{Highlighting}
\end{Shaded}

\begin{verbatim}
[1] 100   5
\end{verbatim}

\begin{Shaded}
\begin{Highlighting}[]
\CommentTok{\# 查看資料結構}
\FunctionTok{str}\NormalTok{(my\_data)}
\end{Highlighting}
\end{Shaded}

\begin{verbatim}
'data.frame':   100 obs. of  5 variables:
 $ patient_id: int  1 2 3 4 5 6 7 8 9 10 ...
 $ treatment : chr  "A" "B" "A" "B" ...
 $ age       : int  45 52 38 61 55 42 49 58 36 64 ...
 $ gender    : chr  "M" "F" "M" "F" ...
 $ los       : int  5 8 4 12 6 9 5 11 3 14 ...
\end{verbatim}

\begin{Shaded}
\begin{Highlighting}[]
\CommentTok{\# 查看基本統計摘要}
\FunctionTok{summary}\NormalTok{(my\_data)}
\end{Highlighting}
\end{Shaded}

\begin{verbatim}
   patient_id      treatment              age           gender         
 Min.   :  1.00   Length:100         Min.   :35.00   Length:100        
 1st Qu.: 25.75   Class :character   1st Qu.:41.00   Class :character  
 Median : 50.50   Mode  :character   Median :49.00   Mode  :character  
 Mean   : 50.50                      Mean   :48.77                     
 3rd Qu.: 75.25                      3rd Qu.:56.00                     
 Max.   :100.00                      Max.   :64.00                     
      los       
 Min.   : 3.00  
 1st Qu.: 5.00  
 Median : 7.50  
 Mean   : 8.07  
 3rd Qu.:11.00  
 Max.   :17.00  
\end{verbatim}

\section{任務
8:理解你看到的東西}\label{ux4efbux52d9-8ux7406ux89e3ux4f60ux770bux5230ux7684ux6771ux897f}

📋 \textbf{把 \texttt{summary(my\_data)} 的輸出複製起來,加上這段話貼給
AI:}

\begin{quote}
這是我的資料摘要,請幫我解讀每個欄位的意義,以及有沒有什麼需要注意的地方(例如遺漏值、異常值):
【貼上 summary 輸出】
\end{quote}

\subsection{資料摘要解讀}\label{ux8cc7ux6599ux6458ux8981ux89e3ux8b80}

從 \texttt{summary()} 的輸出,我們可以看到:

\begin{itemize}
\tightlist
\item
  \textbf{patient\_id}: 病人編號,從 1 到 100
\item
  \textbf{treatment}: 治療組別,A 組和 B 組各 50 人
\item
  \textbf{age}: 年齡分佈,平均約 47 歲
\item
  \textbf{gender}: 性別分佈,男女各半
\item
  \textbf{los}: 住院天數,平均約 8 天,B 組似乎住院天數較長
\end{itemize}

\subsection{檢查遺漏值}\label{ux6aa2ux67e5ux907aux6f0fux503c}

\begin{Shaded}
\begin{Highlighting}[]
\CommentTok{\# 檢查是否有遺漏值}
\FunctionTok{sum}\NormalTok{(}\FunctionTok{is.na}\NormalTok{(my\_data))}
\end{Highlighting}
\end{Shaded}

\begin{verbatim}
[1] 0
\end{verbatim}

\begin{Shaded}
\begin{Highlighting}[]
\CommentTok{\# 查看每個欄位的遺漏值數量}
\FunctionTok{colSums}\NormalTok{(}\FunctionTok{is.na}\NormalTok{(my\_data))}
\end{Highlighting}
\end{Shaded}

\begin{verbatim}
patient_id  treatment        age     gender        los 
         0          0          0          0          0 
\end{verbatim}

\bookmarksetup{startatroot}

\chapter{第三部分:產出你的 Table
1}\label{ux7b2cux4e09ux90e8ux5206ux7522ux51faux4f60ux7684-table-1}

本部分預計時間:40 分鐘

\section{任務 9:Table 1
是什麼?}\label{ux4efbux52d9-9table-1-ux662fux4ec0ux9ebc}

📋 \textbf{複製這段話,貼給 AI:}

\begin{quote}
在醫學論文裡面,Table 1
通常是什麼?它的目的是什麼?裡面通常會放哪些東西?
\end{quote}

\subsection{Table 1 說明}\label{table-1-ux8aaaux660e}

在醫學論文中,Table 1 通常是「基線特徵表」(Baseline Characteristics
Table),用來:

\begin{enumerate}
\def\labelenumi{\arabic{enumi}.}
\tightlist
\item
  展示研究對象的基本特徵
\item
  比較不同組別的基線資料
\item
  讓讀者判斷組別是否平衡
\item
  提供研究族群的整體概況
\end{enumerate}

\section{任務 10:你的第一個 Table
1}\label{ux4efbux52d9-10ux4f60ux7684ux7b2cux4e00ux500b-table-1}

📋 \textbf{複製這段話,貼給 AI:}

\begin{quote}
我有一個 R 資料框叫做 \texttt{my\_data},裡面有這些欄位:

\begin{itemize}
\tightlist
\item
  treatment:治療組別(A 或 B)
\item
  age:年齡
\item
  gender:性別(M 或 F)
\item
  los:住院天數
\end{itemize}

請用 gtsummary 套件幫我做一個 Table 1,依照 treatment
分組,顯示其他變數的描述性統計。請給我可以直接執行的程式碼。
\end{quote}

\subsection{建立 Table 1}\label{ux5efaux7acb-table-1}

\begin{Shaded}
\begin{Highlighting}[]
\FunctionTok{library}\NormalTok{(gtsummary)}
\FunctionTok{library}\NormalTok{(dplyr)}

\CommentTok{\# 讀取資料}
\NormalTok{my\_data }\OtherTok{\textless{}{-}} \FunctionTok{read.csv}\NormalTok{(}\StringTok{"patient\_data.csv"}\NormalTok{)}

\CommentTok{\# 建立基本的 Table 1}
\NormalTok{table1 }\OtherTok{\textless{}{-}}\NormalTok{ my\_data }\SpecialCharTok{\%\textgreater{}\%}
  \FunctionTok{select}\NormalTok{(treatment, age, gender, los) }\SpecialCharTok{\%\textgreater{}\%}
  \FunctionTok{tbl\_summary}\NormalTok{(}
    \AttributeTok{by =}\NormalTok{ treatment,}
    \AttributeTok{statistic =} \FunctionTok{list}\NormalTok{(}
      \FunctionTok{all\_continuous}\NormalTok{() }\SpecialCharTok{\textasciitilde{}} \StringTok{"\{mean\} (\{sd\})"}\NormalTok{,}
      \FunctionTok{all\_categorical}\NormalTok{() }\SpecialCharTok{\textasciitilde{}} \StringTok{"\{n\} (\{p\}\%)"}
\NormalTok{    )}
\NormalTok{  )}

\NormalTok{table1}
\end{Highlighting}
\end{Shaded}

\begin{table}
\fontsize{12.0pt}{14.0pt}\selectfont
\begin{tabular*}{\linewidth}{@{\extracolsep{\fill}}lcc}
\toprule
\textbf{Characteristic} & \textbf{A}  N = 50\textsuperscript{\textit{1}} & \textbf{B}  N = 50\textsuperscript{\textit{1}} \\ 
\midrule\addlinespace[2.5pt]
age & 42 (4) & 56 (5) \\ 
gender &  &  \\ 
    F & 23 (46\%) & 27 (54\%) \\ 
    M & 27 (54\%) & 23 (46\%) \\ 
los & 5 (1) & 12 (2) \\ 
\bottomrule
\end{tabular*}
\begin{minipage}{\linewidth}
\textsuperscript{\textit{1}}Mean (SD); n (\%)\\
\end{minipage}
\end{table}

\section{任務
11:加上統計檢定}\label{ux4efbux52d9-11ux52a0ux4e0aux7d71ux8a08ux6aa2ux5b9a}

📋 \textbf{複製這段話,貼給 AI:}

\begin{quote}
剛剛的 Table 1 很棒。請幫我加上
p-value,讓我可以看出兩組之間有沒有統計顯著差異。
\end{quote}

\subsection{加入 p-value}\label{ux52a0ux5165-p-value}

\begin{Shaded}
\begin{Highlighting}[]
\CommentTok{\# 建立含 p{-}value 的 Table 1}
\NormalTok{table1\_with\_p }\OtherTok{\textless{}{-}}\NormalTok{ my\_data }\SpecialCharTok{\%\textgreater{}\%}
  \FunctionTok{select}\NormalTok{(treatment, age, gender, los) }\SpecialCharTok{\%\textgreater{}\%}
  \FunctionTok{tbl\_summary}\NormalTok{(}
    \AttributeTok{by =}\NormalTok{ treatment,}
    \AttributeTok{statistic =} \FunctionTok{list}\NormalTok{(}
      \FunctionTok{all\_continuous}\NormalTok{() }\SpecialCharTok{\textasciitilde{}} \StringTok{"\{mean\} (\{sd\})"}\NormalTok{,}
      \FunctionTok{all\_categorical}\NormalTok{() }\SpecialCharTok{\textasciitilde{}} \StringTok{"\{n\} (\{p\}\%)"}
\NormalTok{    )}
\NormalTok{  ) }\SpecialCharTok{\%\textgreater{}\%}
  \FunctionTok{add\_p}\NormalTok{()}

\NormalTok{table1\_with\_p}
\end{Highlighting}
\end{Shaded}

\begin{table}
\fontsize{12.0pt}{14.0pt}\selectfont
\begin{tabular*}{\linewidth}{@{\extracolsep{\fill}}lccc}
\toprule
\textbf{Characteristic} & \textbf{A}  N = 50\textsuperscript{\textit{1}} & \textbf{B}  N = 50\textsuperscript{\textit{1}} & \textbf{p-value}\textsuperscript{\textit{2}} \\ 
\midrule\addlinespace[2.5pt]
age & 42 (4) & 56 (5) & <0.001 \\ 
gender &  &  & 0.4 \\ 
    F & 23 (46\%) & 27 (54\%) &  \\ 
    M & 27 (54\%) & 23 (46\%) &  \\ 
los & 5 (1) & 12 (2) & <0.001 \\ 
\bottomrule
\end{tabular*}
\begin{minipage}{\linewidth}
\textsuperscript{\textit{1}}Mean (SD); n (\%)\\
\textsuperscript{\textit{2}}Wilcoxon rank sum test; Pearson's Chi-squared test\\
\end{minipage}
\end{table}

\section{任務 12:看懂 gtsummary
的選擇}\label{ux4efbux52d9-12ux770bux61c2-gtsummary-ux7684ux9078ux64c7}

📋 \textbf{複製這段話,貼給 AI:}

\begin{quote}
gtsummary 在計算 p-value
的時候,是怎麼決定要用什麼統計方法的?例如,什麼時候用
t-test、什麼時候用 Wilcoxon、什麼時候用卡方檢定?
\end{quote}

\subsection{gtsummary
的統計方法選擇}\label{gtsummary-ux7684ux7d71ux8a08ux65b9ux6cd5ux9078ux64c7}

gtsummary 會根據資料類型自動選擇適當的統計檢定:

\begin{itemize}
\tightlist
\item
  \textbf{連續變數}:預設使用 Wilcoxon rank-sum test(無母數檢定)
\item
  \textbf{類別變數}:使用 Chi-square test(卡方檢定)或 Fisher's exact
  test(當樣本數小時)
\end{itemize}

你可以自訂統計方法:

\begin{Shaded}
\begin{Highlighting}[]
\CommentTok{\# 自訂統計檢定方法}
\NormalTok{table1\_custom }\OtherTok{\textless{}{-}}\NormalTok{ my\_data }\SpecialCharTok{\%\textgreater{}\%}
  \FunctionTok{select}\NormalTok{(treatment, age, gender, los) }\SpecialCharTok{\%\textgreater{}\%}
  \FunctionTok{tbl\_summary}\NormalTok{(}
    \AttributeTok{by =}\NormalTok{ treatment,}
    \AttributeTok{statistic =} \FunctionTok{list}\NormalTok{(}
      \FunctionTok{all\_continuous}\NormalTok{() }\SpecialCharTok{\textasciitilde{}} \StringTok{"\{mean\} (\{sd\})"}\NormalTok{,}
      \FunctionTok{all\_categorical}\NormalTok{() }\SpecialCharTok{\textasciitilde{}} \StringTok{"\{n\} (\{p\}\%)"}
\NormalTok{    )}
\NormalTok{  ) }\SpecialCharTok{\%\textgreater{}\%}
  \FunctionTok{add\_p}\NormalTok{(}
    \AttributeTok{test =} \FunctionTok{list}\NormalTok{(}
\NormalTok{      age }\SpecialCharTok{\textasciitilde{}} \StringTok{"t.test"}\NormalTok{,        }\CommentTok{\# 使用 t{-}test}
\NormalTok{      los }\SpecialCharTok{\textasciitilde{}} \StringTok{"wilcox.test"}\NormalTok{,   }\CommentTok{\# 使用 Wilcoxon test}
\NormalTok{      gender }\SpecialCharTok{\textasciitilde{}} \StringTok{"chisq.test"}  \CommentTok{\# 使用卡方檢定}
\NormalTok{    )}
\NormalTok{  )}

\NormalTok{table1\_custom}
\end{Highlighting}
\end{Shaded}

\begin{table}
\fontsize{12.0pt}{14.0pt}\selectfont
\begin{tabular*}{\linewidth}{@{\extracolsep{\fill}}lccc}
\toprule
\textbf{Characteristic} & \textbf{A}  N = 50\textsuperscript{\textit{1}} & \textbf{B}  N = 50\textsuperscript{\textit{1}} & \textbf{p-value}\textsuperscript{\textit{2}} \\ 
\midrule\addlinespace[2.5pt]
age & 42 (4) & 56 (5) & <0.001 \\ 
gender &  &  & 0.5 \\ 
    F & 23 (46\%) & 27 (54\%) &  \\ 
    M & 27 (54\%) & 23 (46\%) &  \\ 
los & 5 (1) & 12 (2) & <0.001 \\ 
\bottomrule
\end{tabular*}
\begin{minipage}{\linewidth}
\textsuperscript{\textit{1}}Mean (SD); n (\%)\\
\textsuperscript{\textit{2}}Welch Two Sample t-test; Pearson's Chi-squared test; Wilcoxon rank sum test\\
\end{minipage}
\end{table}

\section{任務
13:客製化你的表格}\label{ux4efbux52d9-13ux5ba2ux88fdux5316ux4f60ux7684ux8868ux683c}

📋 \textbf{複製這段話,貼給 AI:}

\begin{quote}
我想要修改我的 gtsummary 表格:

\begin{enumerate}
\def\labelenumi{\arabic{enumi}.}
\tightlist
\item
  連續變數顯示「平均值 ± 標準差」而不是中位數
\item
  把欄位名稱改成中文(age → 年齡、gender → 性別、los → 住院天數)
\item
  p-value 如果小於 0.05 就用粗體標示
\end{enumerate}
\end{quote}

\subsection{客製化表格}\label{ux5ba2ux88fdux5316ux8868ux683c}

\begin{Shaded}
\begin{Highlighting}[]
\CommentTok{\# 客製化 Table 1}
\NormalTok{table1\_final }\OtherTok{\textless{}{-}}\NormalTok{ my\_data }\SpecialCharTok{\%\textgreater{}\%}
  \FunctionTok{select}\NormalTok{(treatment, age, gender, los) }\SpecialCharTok{\%\textgreater{}\%}
  \FunctionTok{tbl\_summary}\NormalTok{(}
    \AttributeTok{by =}\NormalTok{ treatment,}
    \AttributeTok{label =} \FunctionTok{list}\NormalTok{(}
\NormalTok{      age }\SpecialCharTok{\textasciitilde{}} \StringTok{"年齡"}\NormalTok{,}
\NormalTok{      gender }\SpecialCharTok{\textasciitilde{}} \StringTok{"性別"}\NormalTok{,}
\NormalTok{      los }\SpecialCharTok{\textasciitilde{}} \StringTok{"住院天數"}
\NormalTok{    ),}
    \AttributeTok{statistic =} \FunctionTok{list}\NormalTok{(}
      \FunctionTok{all\_continuous}\NormalTok{() }\SpecialCharTok{\textasciitilde{}} \StringTok{"\{mean\} ± \{sd\}"}\NormalTok{,}
      \FunctionTok{all\_categorical}\NormalTok{() }\SpecialCharTok{\textasciitilde{}} \StringTok{"\{n\} (\{p\}\%)"}
\NormalTok{    )}
\NormalTok{  ) }\SpecialCharTok{\%\textgreater{}\%}
  \FunctionTok{add\_p}\NormalTok{() }\SpecialCharTok{\%\textgreater{}\%}
  \FunctionTok{bold\_p}\NormalTok{(}\AttributeTok{t =} \FloatTok{0.05}\NormalTok{) }\SpecialCharTok{\%\textgreater{}\%}
  \FunctionTok{modify\_header}\NormalTok{(label }\SpecialCharTok{\textasciitilde{}} \StringTok{"**變項**"}\NormalTok{) }\SpecialCharTok{\%\textgreater{}\%}
  \FunctionTok{modify\_spanning\_header}\NormalTok{(}\FunctionTok{c}\NormalTok{(}\StringTok{"stat\_1"}\NormalTok{, }\StringTok{"stat\_2"}\NormalTok{) }\SpecialCharTok{\textasciitilde{}} \StringTok{"**治療組別**"}\NormalTok{)}

\NormalTok{table1\_final}
\end{Highlighting}
\end{Shaded}

\begin{table}
\fontsize{12.0pt}{14.0pt}\selectfont
\begin{tabular*}{\linewidth}{@{\extracolsep{\fill}}lccc}
\toprule
 & \multicolumn{2}{c}{\textbf{治療組別}} &  \\ 
\cmidrule(lr){2-3}
\textbf{變項} & \textbf{A}  N = 50\textsuperscript{\textit{1}} & \textbf{B}  N = 50\textsuperscript{\textit{1}} & \textbf{p-value}\textsuperscript{\textit{2}} \\ 
\midrule\addlinespace[2.5pt]
年齡 & 42 \ensuremath{\pm} 4 & 56 \ensuremath{\pm} 5 & {\bfseries <0.001} \\ 
性別 &  &  & 0.4 \\ 
    F & 23 (46\%) & 27 (54\%) &  \\ 
    M & 27 (54\%) & 23 (46\%) &  \\ 
住院天數 & 5 \ensuremath{\pm} 1 & 12 \ensuremath{\pm} 2 & {\bfseries <0.001} \\ 
\bottomrule
\end{tabular*}
\begin{minipage}{\linewidth}
\textsuperscript{\textit{1}}Mean ± SD; n (\%)\\
\textsuperscript{\textit{2}}Wilcoxon rank sum test; Pearson's Chi-squared test\\
\end{minipage}
\end{table}

\section{任務
14:匯出你的表格}\label{ux4efbux52d9-14ux532fux51faux4f60ux7684ux8868ux683c}

📋 \textbf{複製這段話,貼給 AI:}

\begin{quote}
我做好了一個 gtsummary 的表格,存在變數 \texttt{my\_table}
裡面。我想要把它輸出成 Word 檔案,方便貼到我的論文。請給我程式碼。
\end{quote}

\subsection{匯出表格}\label{ux532fux51faux8868ux683c}

\begin{Shaded}
\begin{Highlighting}[]
\CommentTok{\# 安裝必要套件(如果還沒安裝)}
\CommentTok{\# install.packages("flextable")}

\FunctionTok{library}\NormalTok{(flextable)}

\CommentTok{\# 轉換為 flextable 格式並存成 Word 檔}
\NormalTok{table1\_final }\SpecialCharTok{\%\textgreater{}\%}
  \FunctionTok{as\_flex\_table}\NormalTok{() }\SpecialCharTok{\%\textgreater{}\%}
\NormalTok{  flextable}\SpecialCharTok{::}\FunctionTok{save\_as\_docx}\NormalTok{(}\AttributeTok{path =} \StringTok{"Table1.docx"}\NormalTok{)}

\FunctionTok{print}\NormalTok{(}\StringTok{"Table 1 已成功匯出為 Table1.docx"}\NormalTok{)}
\end{Highlighting}
\end{Shaded}

\bookmarksetup{startatroot}

\chapter{第四部分:畫出論文等級的圖}\label{ux7b2cux56dbux90e8ux5206ux756bux51faux8ad6ux6587ux7b49ux7d1aux7684ux5716}

本部分預計時間:40 分鐘

\section{任務
15:第一個盒狀圖}\label{ux4efbux52d9-15ux7b2cux4e00ux500bux76d2ux72c0ux5716}

📋 \textbf{複製這段話,貼給 AI:}

\begin{quote}
我有一個資料框 \texttt{my\_data},裡面有 treatment(A 或 B 兩組)和
los(住院天數)。請用 ggplot2 畫一個盒狀圖,比較兩組的住院天數分佈。
\end{quote}

\subsection{基本盒狀圖}\label{ux57faux672cux76d2ux72c0ux5716}

\begin{Shaded}
\begin{Highlighting}[]
\FunctionTok{library}\NormalTok{(ggplot2)}

\CommentTok{\# 讀取資料}
\NormalTok{my\_data }\OtherTok{\textless{}{-}} \FunctionTok{read.csv}\NormalTok{(}\StringTok{"patient\_data.csv"}\NormalTok{)}

\CommentTok{\# 畫基本盒狀圖}
\FunctionTok{ggplot}\NormalTok{(my\_data, }\FunctionTok{aes}\NormalTok{(}\AttributeTok{x =}\NormalTok{ treatment, }\AttributeTok{y =}\NormalTok{ los)) }\SpecialCharTok{+}
  \FunctionTok{geom\_boxplot}\NormalTok{()}
\end{Highlighting}
\end{Shaded}

\pandocbounded{\includegraphics[keepaspectratio]{part4_files/figure-pdf/boxplot-basic-1.pdf}}

\section{任務
16:讓圖變專業}\label{ux4efbux52d9-16ux8b93ux5716ux8b8aux5c08ux696d}

📋 \textbf{複製這段話,貼給 AI:}

\begin{quote}
請幫我美化這個盒狀圖:

\begin{enumerate}
\def\labelenumi{\arabic{enumi}.}
\tightlist
\item
  加上標題「兩組治療的住院天數比較」
\item
  X 軸標籤改成「治療組別」,Y 軸改成「住院天數(天)」
\item
  兩組用不同顏色(藍色和橘色)
\item
  使用簡潔的主題風格
\item
  不要顯示圖例(因為 X 軸已經說明了)
\end{enumerate}
\end{quote}

\subsection{專業化盒狀圖}\label{ux5c08ux696dux5316ux76d2ux72c0ux5716}

\begin{Shaded}
\begin{Highlighting}[]
\CommentTok{\# 畫專業的盒狀圖}
\FunctionTok{ggplot}\NormalTok{(my\_data, }\FunctionTok{aes}\NormalTok{(}\AttributeTok{x =}\NormalTok{ treatment, }\AttributeTok{y =}\NormalTok{ los, }\AttributeTok{fill =}\NormalTok{ treatment)) }\SpecialCharTok{+}
  \FunctionTok{geom\_boxplot}\NormalTok{() }\SpecialCharTok{+}
  \FunctionTok{scale\_fill\_manual}\NormalTok{(}\AttributeTok{values =} \FunctionTok{c}\NormalTok{(}\StringTok{"A"} \OtherTok{=} \StringTok{"steelblue"}\NormalTok{, }\StringTok{"B"} \OtherTok{=} \StringTok{"darkorange"}\NormalTok{)) }\SpecialCharTok{+}
  \FunctionTok{labs}\NormalTok{(}
    \AttributeTok{title =} \StringTok{"兩組治療的住院天數比較"}\NormalTok{,}
    \AttributeTok{x =} \StringTok{"治療組別"}\NormalTok{,}
    \AttributeTok{y =} \StringTok{"住院天數(天)"}
\NormalTok{  ) }\SpecialCharTok{+}
  \FunctionTok{theme\_minimal}\NormalTok{() }\SpecialCharTok{+}
  \FunctionTok{theme}\NormalTok{(}
    \AttributeTok{legend.position =} \StringTok{"none"}\NormalTok{,}
    \AttributeTok{plot.title =} \FunctionTok{element\_text}\NormalTok{(}\AttributeTok{size =} \DecValTok{14}\NormalTok{, }\AttributeTok{face =} \StringTok{"bold"}\NormalTok{, }\AttributeTok{hjust =} \FloatTok{0.5}\NormalTok{)}
\NormalTok{  )}
\end{Highlighting}
\end{Shaded}

\pandocbounded{\includegraphics[keepaspectratio]{part4_files/figure-pdf/boxplot-professional-1.pdf}}

\section{任務
17:加上資料點}\label{ux4efbux52d9-17ux52a0ux4e0aux8cc7ux6599ux9ede}

📋 \textbf{複製這段話,貼給 AI:}

\begin{quote}
我想在盒狀圖上面疊加個別的資料點,讓讀者可以看到實際的分佈。但資料點不要完全重疊,要有一點水平的隨機散開。請修改程式碼。
\end{quote}

\subsection{加入資料點的盒狀圖}\label{ux52a0ux5165ux8cc7ux6599ux9edeux7684ux76d2ux72c0ux5716}

\begin{Shaded}
\begin{Highlighting}[]
\CommentTok{\# 盒狀圖加上資料點}
\FunctionTok{ggplot}\NormalTok{(my\_data, }\FunctionTok{aes}\NormalTok{(}\AttributeTok{x =}\NormalTok{ treatment, }\AttributeTok{y =}\NormalTok{ los, }\AttributeTok{fill =}\NormalTok{ treatment)) }\SpecialCharTok{+}
  \FunctionTok{geom\_boxplot}\NormalTok{(}\AttributeTok{alpha =} \FloatTok{0.7}\NormalTok{) }\SpecialCharTok{+}
  \FunctionTok{geom\_jitter}\NormalTok{(}\AttributeTok{width =} \FloatTok{0.2}\NormalTok{, }\AttributeTok{alpha =} \FloatTok{0.5}\NormalTok{, }\AttributeTok{size =} \DecValTok{2}\NormalTok{) }\SpecialCharTok{+}
  \FunctionTok{scale\_fill\_manual}\NormalTok{(}\AttributeTok{values =} \FunctionTok{c}\NormalTok{(}\StringTok{"A"} \OtherTok{=} \StringTok{"steelblue"}\NormalTok{, }\StringTok{"B"} \OtherTok{=} \StringTok{"darkorange"}\NormalTok{)) }\SpecialCharTok{+}
  \FunctionTok{labs}\NormalTok{(}
    \AttributeTok{title =} \StringTok{"兩組治療的住院天數比較"}\NormalTok{,}
    \AttributeTok{x =} \StringTok{"治療組別"}\NormalTok{,}
    \AttributeTok{y =} \StringTok{"住院天數(天)"}
\NormalTok{  ) }\SpecialCharTok{+}
  \FunctionTok{theme\_minimal}\NormalTok{() }\SpecialCharTok{+}
  \FunctionTok{theme}\NormalTok{(}
    \AttributeTok{legend.position =} \StringTok{"none"}\NormalTok{,}
    \AttributeTok{plot.title =} \FunctionTok{element\_text}\NormalTok{(}\AttributeTok{size =} \DecValTok{14}\NormalTok{, }\AttributeTok{face =} \StringTok{"bold"}\NormalTok{, }\AttributeTok{hjust =} \FloatTok{0.5}\NormalTok{)}
\NormalTok{  )}
\end{Highlighting}
\end{Shaded}

\pandocbounded{\includegraphics[keepaspectratio]{part4_files/figure-pdf/boxplot-with-points-1.pdf}}

\section{任務
18:其他類型的圖}\label{ux4efbux52d9-18ux5176ux4ed6ux985eux578bux7684ux5716}

\subsection{選項 A:長條圖}\label{ux9078ux9805-aux9577ux689dux5716}

\begin{Shaded}
\begin{Highlighting}[]
\CommentTok{\# 性別分佈長條圖}
\FunctionTok{ggplot}\NormalTok{(my\_data, }\FunctionTok{aes}\NormalTok{(}\AttributeTok{x =}\NormalTok{ gender, }\AttributeTok{fill =}\NormalTok{ gender)) }\SpecialCharTok{+}
  \FunctionTok{geom\_bar}\NormalTok{() }\SpecialCharTok{+}
  \FunctionTok{scale\_fill\_manual}\NormalTok{(}\AttributeTok{values =} \FunctionTok{c}\NormalTok{(}\StringTok{"M"} \OtherTok{=} \StringTok{"lightblue"}\NormalTok{, }\StringTok{"F"} \OtherTok{=} \StringTok{"pink"}\NormalTok{)) }\SpecialCharTok{+}
  \FunctionTok{labs}\NormalTok{(}
    \AttributeTok{title =} \StringTok{"性別分佈"}\NormalTok{,}
    \AttributeTok{x =} \StringTok{"性別"}\NormalTok{,}
    \AttributeTok{y =} \StringTok{"人數"}
\NormalTok{  ) }\SpecialCharTok{+}
  \FunctionTok{theme\_minimal}\NormalTok{() }\SpecialCharTok{+}
  \FunctionTok{theme}\NormalTok{(}\AttributeTok{legend.position =} \StringTok{"none"}\NormalTok{)}
\end{Highlighting}
\end{Shaded}

\pandocbounded{\includegraphics[keepaspectratio]{part4_files/figure-pdf/barplot-1.pdf}}

\subsection{選項 B:散佈圖}\label{ux9078ux9805-bux6563ux4f48ux5716}

\begin{Shaded}
\begin{Highlighting}[]
\CommentTok{\# 年齡與住院天數散佈圖}
\FunctionTok{ggplot}\NormalTok{(my\_data, }\FunctionTok{aes}\NormalTok{(}\AttributeTok{x =}\NormalTok{ age, }\AttributeTok{y =}\NormalTok{ los, }\AttributeTok{color =}\NormalTok{ treatment)) }\SpecialCharTok{+}
  \FunctionTok{geom\_point}\NormalTok{(}\AttributeTok{size =} \DecValTok{3}\NormalTok{, }\AttributeTok{alpha =} \FloatTok{0.6}\NormalTok{) }\SpecialCharTok{+}
  \FunctionTok{geom\_smooth}\NormalTok{(}\AttributeTok{method =} \StringTok{"lm"}\NormalTok{, }\AttributeTok{se =} \ConstantTok{TRUE}\NormalTok{) }\SpecialCharTok{+}
  \FunctionTok{scale\_color\_manual}\NormalTok{(}\AttributeTok{values =} \FunctionTok{c}\NormalTok{(}\StringTok{"A"} \OtherTok{=} \StringTok{"steelblue"}\NormalTok{, }\StringTok{"B"} \OtherTok{=} \StringTok{"darkorange"}\NormalTok{)) }\SpecialCharTok{+}
  \FunctionTok{labs}\NormalTok{(}
    \AttributeTok{title =} \StringTok{"年齡與住院天數的關係"}\NormalTok{,}
    \AttributeTok{x =} \StringTok{"年齡(歲)"}\NormalTok{,}
    \AttributeTok{y =} \StringTok{"住院天數(天)"}\NormalTok{,}
    \AttributeTok{color =} \StringTok{"治療組別"}
\NormalTok{  ) }\SpecialCharTok{+}
  \FunctionTok{theme\_minimal}\NormalTok{()}
\end{Highlighting}
\end{Shaded}

\pandocbounded{\includegraphics[keepaspectratio]{part4_files/figure-pdf/scatterplot-1.pdf}}

\subsection{選項 C:直方圖}\label{ux9078ux9805-cux76f4ux65b9ux5716}

\begin{Shaded}
\begin{Highlighting}[]
\CommentTok{\# 住院天數分佈直方圖}
\FunctionTok{ggplot}\NormalTok{(my\_data, }\FunctionTok{aes}\NormalTok{(}\AttributeTok{x =}\NormalTok{ los, }\AttributeTok{fill =}\NormalTok{ treatment)) }\SpecialCharTok{+}
  \FunctionTok{geom\_histogram}\NormalTok{(}\AttributeTok{binwidth =} \DecValTok{2}\NormalTok{, }\AttributeTok{alpha =} \FloatTok{0.7}\NormalTok{, }\AttributeTok{position =} \StringTok{"dodge"}\NormalTok{) }\SpecialCharTok{+}
  \FunctionTok{scale\_fill\_manual}\NormalTok{(}\AttributeTok{values =} \FunctionTok{c}\NormalTok{(}\StringTok{"A"} \OtherTok{=} \StringTok{"steelblue"}\NormalTok{, }\StringTok{"B"} \OtherTok{=} \StringTok{"darkorange"}\NormalTok{)) }\SpecialCharTok{+}
  \FunctionTok{labs}\NormalTok{(}
    \AttributeTok{title =} \StringTok{"住院天數分佈"}\NormalTok{,}
    \AttributeTok{x =} \StringTok{"住院天數(天)"}\NormalTok{,}
    \AttributeTok{y =} \StringTok{"人數"}\NormalTok{,}
    \AttributeTok{fill =} \StringTok{"治療組別"}
\NormalTok{  ) }\SpecialCharTok{+}
  \FunctionTok{theme\_minimal}\NormalTok{()}
\end{Highlighting}
\end{Shaded}

\pandocbounded{\includegraphics[keepaspectratio]{part4_files/figure-pdf/histogram-1.pdf}}

\section{任務
19:存檔你的圖}\label{ux4efbux52d9-19ux5b58ux6a94ux4f60ux7684ux5716}

📋 \textbf{複製這段話,貼給 AI:}

\begin{quote}
我畫好了一張 ggplot2 的圖,想要存成 PNG
檔案,解析度要夠高可以放在論文裡(300 dpi),大小大約是 8 x 6
英吋。請給我存檔的程式碼。
\end{quote}

\subsection{儲存高品質圖檔}\label{ux5132ux5b58ux9ad8ux54c1ux8ceaux5716ux6a94}

\begin{Shaded}
\begin{Highlighting}[]
\CommentTok{\# 儲存最後畫的圖}
\FunctionTok{ggsave}\NormalTok{(}
  \AttributeTok{filename =} \StringTok{"boxplot\_comparison.png"}\NormalTok{,}
  \AttributeTok{width =} \DecValTok{8}\NormalTok{,}
  \AttributeTok{height =} \DecValTok{6}\NormalTok{,}
  \AttributeTok{dpi =} \DecValTok{300}\NormalTok{,}
  \AttributeTok{units =} \StringTok{"in"}
\NormalTok{)}

\FunctionTok{print}\NormalTok{(}\StringTok{"圖表已成功儲存為 boxplot\_comparison.png"}\NormalTok{)}
\end{Highlighting}
\end{Shaded}

\bookmarksetup{startatroot}

\chapter{第五部分:統計檢定}\label{ux7b2cux4e94ux90e8ux5206ux7d71ux8a08ux6aa2ux5b9a}

本部分預計時間:30 分鐘

\section{任務 20:讓 AI
選擇統計方法}\label{ux4efbux52d9-20ux8b93-ai-ux9078ux64c7ux7d71ux8a08ux65b9ux6cd5}

📋 \textbf{複製這段話,貼給 AI:}

\begin{quote}
我想要比較兩組病人(treatment A vs
B)的住院天數(los)有沒有顯著差異。我的資料存在 \texttt{my\_data}
裡面。

請幫我:

\begin{enumerate}
\def\labelenumi{\arabic{enumi}.}
\tightlist
\item
  判斷應該用什麼統計方法(為什麼)
\item
  給我執行這個檢定的 R 程式碼
\item
  告訴我怎麼解讀結果
\end{enumerate}
\end{quote}

\subsection{統計檢定選擇}\label{ux7d71ux8a08ux6aa2ux5b9aux9078ux64c7}

\begin{Shaded}
\begin{Highlighting}[]
\CommentTok{\# 讀取資料}
\NormalTok{my\_data }\OtherTok{\textless{}{-}} \FunctionTok{read.csv}\NormalTok{(}\StringTok{"patient\_data.csv"}\NormalTok{)}

\CommentTok{\# 先檢查資料分佈}
\FunctionTok{library}\NormalTok{(ggplot2)}

\CommentTok{\# 檢視分佈圖}
\FunctionTok{ggplot}\NormalTok{(my\_data, }\FunctionTok{aes}\NormalTok{(}\AttributeTok{x =}\NormalTok{ los, }\AttributeTok{fill =}\NormalTok{ treatment)) }\SpecialCharTok{+}
  \FunctionTok{geom\_histogram}\NormalTok{(}\AttributeTok{binwidth =} \DecValTok{1}\NormalTok{, }\AttributeTok{alpha =} \FloatTok{0.6}\NormalTok{, }\AttributeTok{position =} \StringTok{"dodge"}\NormalTok{) }\SpecialCharTok{+}
  \FunctionTok{facet\_wrap}\NormalTok{(}\SpecialCharTok{\textasciitilde{}}\NormalTok{treatment, }\AttributeTok{ncol =} \DecValTok{1}\NormalTok{) }\SpecialCharTok{+}
  \FunctionTok{theme\_minimal}\NormalTok{()}
\end{Highlighting}
\end{Shaded}

\pandocbounded{\includegraphics[keepaspectratio]{part5_files/figure-pdf/statistical-test-1.pdf}}

\begin{Shaded}
\begin{Highlighting}[]
\CommentTok{\# 常態性檢定}
\NormalTok{shapiro\_a }\OtherTok{\textless{}{-}} \FunctionTok{shapiro.test}\NormalTok{(my\_data}\SpecialCharTok{$}\NormalTok{los[my\_data}\SpecialCharTok{$}\NormalTok{treatment }\SpecialCharTok{==} \StringTok{"A"}\NormalTok{])}
\NormalTok{shapiro\_b }\OtherTok{\textless{}{-}} \FunctionTok{shapiro.test}\NormalTok{(my\_data}\SpecialCharTok{$}\NormalTok{los[my\_data}\SpecialCharTok{$}\NormalTok{treatment }\SpecialCharTok{==} \StringTok{"B"}\NormalTok{])}

\FunctionTok{print}\NormalTok{(}\StringTok{"Shapiro{-}Wilk 常態性檢定結果:"}\NormalTok{)}
\end{Highlighting}
\end{Shaded}

\begin{verbatim}
[1] "Shapiro-Wilk 常態性檢定結果:"
\end{verbatim}

\begin{Shaded}
\begin{Highlighting}[]
\FunctionTok{print}\NormalTok{(}\FunctionTok{paste}\NormalTok{(}\StringTok{"A 組 p{-}value:"}\NormalTok{, }\FunctionTok{round}\NormalTok{(shapiro\_a}\SpecialCharTok{$}\NormalTok{p.value, }\DecValTok{4}\NormalTok{)))}
\end{Highlighting}
\end{Shaded}

\begin{verbatim}
[1] "A 組 p-value: 0.0004"
\end{verbatim}

\begin{Shaded}
\begin{Highlighting}[]
\FunctionTok{print}\NormalTok{(}\FunctionTok{paste}\NormalTok{(}\StringTok{"B 組 p{-}value:"}\NormalTok{, }\FunctionTok{round}\NormalTok{(shapiro\_b}\SpecialCharTok{$}\NormalTok{p.value, }\DecValTok{4}\NormalTok{)))}
\end{Highlighting}
\end{Shaded}

\begin{verbatim}
[1] "B 組 p-value: 0.0829"
\end{verbatim}

\subsection{執行統計檢定}\label{ux57f7ux884cux7d71ux8a08ux6aa2ux5b9a}

\begin{Shaded}
\begin{Highlighting}[]
\CommentTok{\# 執行 t{-}test(假設資料符合常態分佈)}
\NormalTok{t\_test\_result }\OtherTok{\textless{}{-}} \FunctionTok{t.test}\NormalTok{(los }\SpecialCharTok{\textasciitilde{}}\NormalTok{ treatment, }\AttributeTok{data =}\NormalTok{ my\_data)}
\FunctionTok{print}\NormalTok{(t\_test\_result)}
\end{Highlighting}
\end{Shaded}

\begin{verbatim}

    Welch Two Sample t-test

data:  los by treatment
t = -19.12, df = 68.524, p-value < 2.2e-16
alternative hypothesis: true difference in means between group A and group B is not equal to 0
95 percent confidence interval:
 -7.708356 -6.251644
sample estimates:
mean in group A mean in group B 
           4.58           11.56 
\end{verbatim}

\begin{Shaded}
\begin{Highlighting}[]
\CommentTok{\# 執行 Wilcoxon rank{-}sum test(無母數檢定)}
\NormalTok{wilcox\_result }\OtherTok{\textless{}{-}} \FunctionTok{wilcox.test}\NormalTok{(los }\SpecialCharTok{\textasciitilde{}}\NormalTok{ treatment, }\AttributeTok{data =}\NormalTok{ my\_data)}
\FunctionTok{print}\NormalTok{(wilcox\_result)}
\end{Highlighting}
\end{Shaded}

\begin{verbatim}

    Wilcoxon rank sum test with continuity correction

data:  los by treatment
W = 0, p-value < 2.2e-16
alternative hypothesis: true location shift is not equal to 0
\end{verbatim}

\section{任務
21:解讀你的結果}\label{ux4efbux52d9-21ux89e3ux8b80ux4f60ux7684ux7d50ux679c}

📋 \textbf{把統計檢定的輸出複製起來,加上這段話貼給 AI:}

\begin{quote}
這是我的統計檢定結果: 【貼上輸出】

請用白話文解釋這個結果,包括:

\begin{enumerate}
\def\labelenumi{\arabic{enumi}.}
\tightlist
\item
  兩組有沒有顯著差異
\item
  p-value 是什麼意思
\item
  這個結果在臨床上可能代表什麼
\end{enumerate}

另外,請幫我:

\begin{enumerate}
\def\labelenumi{\arabic{enumi}.}
\tightlist
\item
  用 broom 套件的 tidy() 函數把統計結果整理成乾淨的表格
\item
  用 capture.output() 或 sink() 把所有結果儲存成純文字檔 report.txt
\item
  產出一份完整的分析報告,包含描述性統計、檢定結果、效應量
\end{enumerate}
\end{quote}

\subsection{結果解讀}\label{ux7d50ux679cux89e3ux8b80}

\begin{Shaded}
\begin{Highlighting}[]
\CommentTok{\# 載入必要套件}
\FunctionTok{library}\NormalTok{(dplyr)}

\CommentTok{\# 安裝並載入 broom 套件(用於整理統計結果)}
\ControlFlowTok{if}\NormalTok{ (}\SpecialCharTok{!}\FunctionTok{require}\NormalTok{(broom, }\AttributeTok{quietly =} \ConstantTok{TRUE}\NormalTok{)) \{}
  \FunctionTok{install.packages}\NormalTok{(}\StringTok{"broom"}\NormalTok{, }\AttributeTok{repos =} \StringTok{"https://cloud.r{-}project.org/"}\NormalTok{)}
\NormalTok{\}}
\FunctionTok{library}\NormalTok{(broom)}

\CommentTok{\# 安裝並載入 effsize 套件(用於計算效應量)}
\ControlFlowTok{if}\NormalTok{ (}\SpecialCharTok{!}\FunctionTok{require}\NormalTok{(effsize, }\AttributeTok{quietly =} \ConstantTok{TRUE}\NormalTok{)) \{}
  \FunctionTok{install.packages}\NormalTok{(}\StringTok{"effsize"}\NormalTok{, }\AttributeTok{repos =} \StringTok{"https://cloud.r{-}project.org/"}\NormalTok{)}
\NormalTok{\}}
\FunctionTok{library}\NormalTok{(effsize)}

\CommentTok{\# 整理結果摘要}
\NormalTok{result\_summary }\OtherTok{\textless{}{-}}\NormalTok{ my\_data }\SpecialCharTok{\%\textgreater{}\%}
  \FunctionTok{group\_by}\NormalTok{(treatment) }\SpecialCharTok{\%\textgreater{}\%}
  \FunctionTok{summarise}\NormalTok{(}
    \AttributeTok{n =} \FunctionTok{n}\NormalTok{(),}
    \AttributeTok{mean\_los =} \FunctionTok{mean}\NormalTok{(los),}
    \AttributeTok{sd\_los =} \FunctionTok{sd}\NormalTok{(los),}
    \AttributeTok{median\_los =} \FunctionTok{median}\NormalTok{(los),}
    \AttributeTok{min\_los =} \FunctionTok{min}\NormalTok{(los),}
    \AttributeTok{max\_los =} \FunctionTok{max}\NormalTok{(los)}
\NormalTok{  )}

\FunctionTok{print}\NormalTok{(}\StringTok{"各組描述性統計:"}\NormalTok{)}
\end{Highlighting}
\end{Shaded}

\begin{verbatim}
[1] "各組描述性統計:"
\end{verbatim}

\begin{Shaded}
\begin{Highlighting}[]
\FunctionTok{print}\NormalTok{(result\_summary)}
\end{Highlighting}
\end{Shaded}

\begin{verbatim}
# A tibble: 2 x 7
  treatment     n mean_los sd_los median_los min_los max_los
  <chr>     <int>    <dbl>  <dbl>      <dbl>   <int>   <int>
1 A            50     4.58   1.07          5       3       7
2 B            50    11.6    2.35         11       8      17
\end{verbatim}

\begin{Shaded}
\begin{Highlighting}[]
\CommentTok{\# 使用 broom::tidy() 整理 t{-}test 結果}
\NormalTok{t\_test\_tidy }\OtherTok{\textless{}{-}} \FunctionTok{tidy}\NormalTok{(t\_test\_result)}
\FunctionTok{print}\NormalTok{(}\StringTok{"T{-}test 結果(broom 整理後):"}\NormalTok{)}
\end{Highlighting}
\end{Shaded}

\begin{verbatim}
[1] "T-test 結果(broom 整理後):"
\end{verbatim}

\begin{Shaded}
\begin{Highlighting}[]
\FunctionTok{print}\NormalTok{(t\_test\_tidy)}
\end{Highlighting}
\end{Shaded}

\begin{verbatim}
# A tibble: 1 x 10
  estimate estimate1 estimate2 statistic  p.value parameter conf.low conf.high
     <dbl>     <dbl>     <dbl>     <dbl>    <dbl>     <dbl>    <dbl>     <dbl>
1    -6.98      4.58      11.6     -19.1 3.54e-29      68.5    -7.71     -6.25
# i 2 more variables: method <chr>, alternative <chr>
\end{verbatim}

\begin{Shaded}
\begin{Highlighting}[]
\CommentTok{\# 使用 broom::tidy() 整理 Wilcoxon 結果}
\NormalTok{wilcox\_tidy }\OtherTok{\textless{}{-}} \FunctionTok{tidy}\NormalTok{(wilcox\_result)}
\FunctionTok{print}\NormalTok{(}\StringTok{"Wilcoxon test 結果(broom 整理後):"}\NormalTok{)}
\end{Highlighting}
\end{Shaded}

\begin{verbatim}
[1] "Wilcoxon test 結果(broom 整理後):"
\end{verbatim}

\begin{Shaded}
\begin{Highlighting}[]
\FunctionTok{print}\NormalTok{(wilcox\_tidy)}
\end{Highlighting}
\end{Shaded}

\begin{verbatim}
# A tibble: 1 x 4
  statistic  p.value method                                          alternative
      <dbl>    <dbl> <chr>                                           <chr>      
1         0 4.63e-18 Wilcoxon rank sum test with continuity correct~ two.sided  
\end{verbatim}

\begin{Shaded}
\begin{Highlighting}[]
\CommentTok{\# 計算效應量(Cohen\textquotesingle{}s d)}
\NormalTok{cohen\_d }\OtherTok{\textless{}{-}} \FunctionTok{cohen.d}\NormalTok{(los }\SpecialCharTok{\textasciitilde{}}\NormalTok{ treatment, }\AttributeTok{data =}\NormalTok{ my\_data)}
\FunctionTok{print}\NormalTok{(}\FunctionTok{paste}\NormalTok{(}\StringTok{"效應量 (Cohen\textquotesingle{}s d):"}\NormalTok{, }\FunctionTok{round}\NormalTok{(cohen\_d}\SpecialCharTok{$}\NormalTok{estimate, }\DecValTok{3}\NormalTok{)))}
\end{Highlighting}
\end{Shaded}

\begin{verbatim}
[1] "效應量 (Cohen's d): -3.824"
\end{verbatim}

\begin{Shaded}
\begin{Highlighting}[]
\CommentTok{\# 解釋}
\ControlFlowTok{if}\NormalTok{ (t\_test\_result}\SpecialCharTok{$}\NormalTok{p.value }\SpecialCharTok{\textless{}} \FloatTok{0.05}\NormalTok{) \{}
  \FunctionTok{print}\NormalTok{(}\StringTok{"結論:兩組在統計上有顯著差異 (p \textless{} 0.05)"}\NormalTok{)}
  \FunctionTok{print}\NormalTok{(}\FunctionTok{paste}\NormalTok{(}\StringTok{"B 組的平均住院天數比 A 組多約"}\NormalTok{,}
              \FunctionTok{round}\NormalTok{(}\FunctionTok{diff}\NormalTok{(result\_summary}\SpecialCharTok{$}\NormalTok{mean\_los), }\DecValTok{1}\NormalTok{), }\StringTok{"天"}\NormalTok{))}
\NormalTok{\} }\ControlFlowTok{else}\NormalTok{ \{}
  \FunctionTok{print}\NormalTok{(}\StringTok{"結論:兩組在統計上無顯著差異 (p \textgreater{}= 0.05)"}\NormalTok{)}
\NormalTok{\}}
\end{Highlighting}
\end{Shaded}

\begin{verbatim}
[1] "結論:兩組在統計上有顯著差異 (p < 0.05)"
[1] "B 組的平均住院天數比 A 組多約 7 天"
\end{verbatim}

\subsection{產出純文字報告}\label{ux7522ux51faux7d14ux6587ux5b57ux5831ux544a}

\begin{Shaded}
\begin{Highlighting}[]
\CommentTok{\# 方法一:使用 sink() 將所有輸出導向檔案}
\FunctionTok{sink}\NormalTok{(}\StringTok{"report.txt"}\NormalTok{)}

\FunctionTok{cat}\NormalTok{(}\StringTok{"="} \SpecialCharTok{\%\textgreater{}\%} \FunctionTok{rep}\NormalTok{(}\DecValTok{60}\NormalTok{) }\SpecialCharTok{\%\textgreater{}\%} \FunctionTok{paste}\NormalTok{(}\AttributeTok{collapse =} \StringTok{""}\NormalTok{), }\StringTok{"}\SpecialCharTok{\textbackslash{}n}\StringTok{"}\NormalTok{)}
\FunctionTok{cat}\NormalTok{(}\StringTok{"        統計分析報告:兩組治療住院天數比較}\SpecialCharTok{\textbackslash{}n}\StringTok{"}\NormalTok{)}
\FunctionTok{cat}\NormalTok{(}\StringTok{"="} \SpecialCharTok{\%\textgreater{}\%} \FunctionTok{rep}\NormalTok{(}\DecValTok{60}\NormalTok{) }\SpecialCharTok{\%\textgreater{}\%} \FunctionTok{paste}\NormalTok{(}\AttributeTok{collapse =} \StringTok{""}\NormalTok{), }\StringTok{"}\SpecialCharTok{\textbackslash{}n}\StringTok{"}\NormalTok{)}
\FunctionTok{cat}\NormalTok{(}\StringTok{"報告產生時間:"}\NormalTok{, }\FunctionTok{format}\NormalTok{(}\FunctionTok{Sys.time}\NormalTok{(), }\StringTok{"\%Y{-}\%m{-}\%d \%H:\%M:\%S"}\NormalTok{), }\StringTok{"}\SpecialCharTok{\textbackslash{}n\textbackslash{}n}\StringTok{"}\NormalTok{)}

\FunctionTok{cat}\NormalTok{(}\StringTok{"{-}"} \SpecialCharTok{\%\textgreater{}\%} \FunctionTok{rep}\NormalTok{(}\DecValTok{60}\NormalTok{) }\SpecialCharTok{\%\textgreater{}\%} \FunctionTok{paste}\NormalTok{(}\AttributeTok{collapse =} \StringTok{""}\NormalTok{), }\StringTok{"}\SpecialCharTok{\textbackslash{}n}\StringTok{"}\NormalTok{)}
\FunctionTok{cat}\NormalTok{(}\StringTok{"一、描述性統計}\SpecialCharTok{\textbackslash{}n}\StringTok{"}\NormalTok{)}
\FunctionTok{cat}\NormalTok{(}\StringTok{"{-}"} \SpecialCharTok{\%\textgreater{}\%} \FunctionTok{rep}\NormalTok{(}\DecValTok{60}\NormalTok{) }\SpecialCharTok{\%\textgreater{}\%} \FunctionTok{paste}\NormalTok{(}\AttributeTok{collapse =} \StringTok{""}\NormalTok{), }\StringTok{"}\SpecialCharTok{\textbackslash{}n}\StringTok{"}\NormalTok{)}
\FunctionTok{print}\NormalTok{(result\_summary)}

\FunctionTok{cat}\NormalTok{(}\StringTok{"}\SpecialCharTok{\textbackslash{}n}\StringTok{"}\NormalTok{)}
\FunctionTok{cat}\NormalTok{(}\StringTok{"{-}"} \SpecialCharTok{\%\textgreater{}\%} \FunctionTok{rep}\NormalTok{(}\DecValTok{60}\NormalTok{) }\SpecialCharTok{\%\textgreater{}\%} \FunctionTok{paste}\NormalTok{(}\AttributeTok{collapse =} \StringTok{""}\NormalTok{), }\StringTok{"}\SpecialCharTok{\textbackslash{}n}\StringTok{"}\NormalTok{)}
\FunctionTok{cat}\NormalTok{(}\StringTok{"二、常態性檢定(Shapiro{-}Wilk Test)}\SpecialCharTok{\textbackslash{}n}\StringTok{"}\NormalTok{)}
\FunctionTok{cat}\NormalTok{(}\StringTok{"{-}"} \SpecialCharTok{\%\textgreater{}\%} \FunctionTok{rep}\NormalTok{(}\DecValTok{60}\NormalTok{) }\SpecialCharTok{\%\textgreater{}\%} \FunctionTok{paste}\NormalTok{(}\AttributeTok{collapse =} \StringTok{""}\NormalTok{), }\StringTok{"}\SpecialCharTok{\textbackslash{}n}\StringTok{"}\NormalTok{)}
\FunctionTok{cat}\NormalTok{(}\StringTok{"A 組 p{-}value:"}\NormalTok{, }\FunctionTok{round}\NormalTok{(shapiro\_a}\SpecialCharTok{$}\NormalTok{p.value, }\DecValTok{4}\NormalTok{), }\StringTok{"}\SpecialCharTok{\textbackslash{}n}\StringTok{"}\NormalTok{)}
\FunctionTok{cat}\NormalTok{(}\StringTok{"B 組 p{-}value:"}\NormalTok{, }\FunctionTok{round}\NormalTok{(shapiro\_b}\SpecialCharTok{$}\NormalTok{p.value, }\DecValTok{4}\NormalTok{), }\StringTok{"}\SpecialCharTok{\textbackslash{}n}\StringTok{"}\NormalTok{)}
\ControlFlowTok{if}\NormalTok{ (shapiro\_a}\SpecialCharTok{$}\NormalTok{p.value }\SpecialCharTok{\textgreater{}=} \FloatTok{0.05} \SpecialCharTok{\&}\NormalTok{ shapiro\_b}\SpecialCharTok{$}\NormalTok{p.value }\SpecialCharTok{\textgreater{}=} \FloatTok{0.05}\NormalTok{) \{}
  \FunctionTok{cat}\NormalTok{(}\StringTok{"結論:兩組資料均符合常態分佈假設}\SpecialCharTok{\textbackslash{}n}\StringTok{"}\NormalTok{)}
\NormalTok{\} }\ControlFlowTok{else}\NormalTok{ \{}
  \FunctionTok{cat}\NormalTok{(}\StringTok{"結論:至少一組資料不符合常態分佈假設}\SpecialCharTok{\textbackslash{}n}\StringTok{"}\NormalTok{)}
\NormalTok{\}}

\FunctionTok{cat}\NormalTok{(}\StringTok{"}\SpecialCharTok{\textbackslash{}n}\StringTok{"}\NormalTok{)}
\FunctionTok{cat}\NormalTok{(}\StringTok{"{-}"} \SpecialCharTok{\%\textgreater{}\%} \FunctionTok{rep}\NormalTok{(}\DecValTok{60}\NormalTok{) }\SpecialCharTok{\%\textgreater{}\%} \FunctionTok{paste}\NormalTok{(}\AttributeTok{collapse =} \StringTok{""}\NormalTok{), }\StringTok{"}\SpecialCharTok{\textbackslash{}n}\StringTok{"}\NormalTok{)}
\FunctionTok{cat}\NormalTok{(}\StringTok{"三、統計檢定結果}\SpecialCharTok{\textbackslash{}n}\StringTok{"}\NormalTok{)}
\FunctionTok{cat}\NormalTok{(}\StringTok{"{-}"} \SpecialCharTok{\%\textgreater{}\%} \FunctionTok{rep}\NormalTok{(}\DecValTok{60}\NormalTok{) }\SpecialCharTok{\%\textgreater{}\%} \FunctionTok{paste}\NormalTok{(}\AttributeTok{collapse =} \StringTok{""}\NormalTok{), }\StringTok{"}\SpecialCharTok{\textbackslash{}n}\StringTok{"}\NormalTok{)}

\FunctionTok{cat}\NormalTok{(}\StringTok{"}\SpecialCharTok{\textbackslash{}n}\StringTok{【T{-}test 結果】}\SpecialCharTok{\textbackslash{}n}\StringTok{"}\NormalTok{)}
\FunctionTok{cat}\NormalTok{(}\StringTok{"檢定方法:"}\NormalTok{, t\_test\_tidy}\SpecialCharTok{$}\NormalTok{method, }\StringTok{"}\SpecialCharTok{\textbackslash{}n}\StringTok{"}\NormalTok{)}
\FunctionTok{cat}\NormalTok{(}\StringTok{"t 統計量:"}\NormalTok{, }\FunctionTok{round}\NormalTok{(t\_test\_tidy}\SpecialCharTok{$}\NormalTok{statistic, }\DecValTok{4}\NormalTok{), }\StringTok{"}\SpecialCharTok{\textbackslash{}n}\StringTok{"}\NormalTok{)}
\FunctionTok{cat}\NormalTok{(}\StringTok{"自由度:"}\NormalTok{, }\FunctionTok{round}\NormalTok{(t\_test\_tidy}\SpecialCharTok{$}\NormalTok{parameter, }\DecValTok{2}\NormalTok{), }\StringTok{"}\SpecialCharTok{\textbackslash{}n}\StringTok{"}\NormalTok{)}
\FunctionTok{cat}\NormalTok{(}\StringTok{"p{-}value:"}\NormalTok{, }\FunctionTok{format}\NormalTok{(t\_test\_tidy}\SpecialCharTok{$}\NormalTok{p.value, }\AttributeTok{digits =} \DecValTok{4}\NormalTok{, }\AttributeTok{scientific =} \ConstantTok{FALSE}\NormalTok{), }\StringTok{"}\SpecialCharTok{\textbackslash{}n}\StringTok{"}\NormalTok{)}
\FunctionTok{cat}\NormalTok{(}\StringTok{"95\% 信賴區間:["}\NormalTok{, }\FunctionTok{round}\NormalTok{(t\_test\_tidy}\SpecialCharTok{$}\NormalTok{conf.low, }\DecValTok{2}\NormalTok{), }\StringTok{", "}\NormalTok{,}
    \FunctionTok{round}\NormalTok{(t\_test\_tidy}\SpecialCharTok{$}\NormalTok{conf.high, }\DecValTok{2}\NormalTok{), }\StringTok{"]}\SpecialCharTok{\textbackslash{}n}\StringTok{"}\NormalTok{)}
\FunctionTok{cat}\NormalTok{(}\StringTok{"兩組平均差異:"}\NormalTok{, }\FunctionTok{round}\NormalTok{(t\_test\_tidy}\SpecialCharTok{$}\NormalTok{estimate1 }\SpecialCharTok{{-}}\NormalTok{ t\_test\_tidy}\SpecialCharTok{$}\NormalTok{estimate2, }\DecValTok{2}\NormalTok{), }\StringTok{"天}\SpecialCharTok{\textbackslash{}n}\StringTok{"}\NormalTok{)}

\FunctionTok{cat}\NormalTok{(}\StringTok{"}\SpecialCharTok{\textbackslash{}n}\StringTok{【Wilcoxon Rank{-}Sum Test 結果】}\SpecialCharTok{\textbackslash{}n}\StringTok{"}\NormalTok{)}
\FunctionTok{cat}\NormalTok{(}\StringTok{"檢定方法:"}\NormalTok{, wilcox\_tidy}\SpecialCharTok{$}\NormalTok{method, }\StringTok{"}\SpecialCharTok{\textbackslash{}n}\StringTok{"}\NormalTok{)}
\FunctionTok{cat}\NormalTok{(}\StringTok{"W 統計量:"}\NormalTok{, wilcox\_tidy}\SpecialCharTok{$}\NormalTok{statistic, }\StringTok{"}\SpecialCharTok{\textbackslash{}n}\StringTok{"}\NormalTok{)}
\FunctionTok{cat}\NormalTok{(}\StringTok{"p{-}value:"}\NormalTok{, }\FunctionTok{format}\NormalTok{(wilcox\_tidy}\SpecialCharTok{$}\NormalTok{p.value, }\AttributeTok{digits =} \DecValTok{4}\NormalTok{, }\AttributeTok{scientific =} \ConstantTok{FALSE}\NormalTok{), }\StringTok{"}\SpecialCharTok{\textbackslash{}n}\StringTok{"}\NormalTok{)}

\FunctionTok{cat}\NormalTok{(}\StringTok{"}\SpecialCharTok{\textbackslash{}n}\StringTok{"}\NormalTok{)}
\FunctionTok{cat}\NormalTok{(}\StringTok{"{-}"} \SpecialCharTok{\%\textgreater{}\%} \FunctionTok{rep}\NormalTok{(}\DecValTok{60}\NormalTok{) }\SpecialCharTok{\%\textgreater{}\%} \FunctionTok{paste}\NormalTok{(}\AttributeTok{collapse =} \StringTok{""}\NormalTok{), }\StringTok{"}\SpecialCharTok{\textbackslash{}n}\StringTok{"}\NormalTok{)}
\FunctionTok{cat}\NormalTok{(}\StringTok{"四、效應量}\SpecialCharTok{\textbackslash{}n}\StringTok{"}\NormalTok{)}
\FunctionTok{cat}\NormalTok{(}\StringTok{"{-}"} \SpecialCharTok{\%\textgreater{}\%} \FunctionTok{rep}\NormalTok{(}\DecValTok{60}\NormalTok{) }\SpecialCharTok{\%\textgreater{}\%} \FunctionTok{paste}\NormalTok{(}\AttributeTok{collapse =} \StringTok{""}\NormalTok{), }\StringTok{"}\SpecialCharTok{\textbackslash{}n}\StringTok{"}\NormalTok{)}
\FunctionTok{cat}\NormalTok{(}\StringTok{"Cohen\textquotesingle{}s d:"}\NormalTok{, }\FunctionTok{round}\NormalTok{(cohen\_d}\SpecialCharTok{$}\NormalTok{estimate, }\DecValTok{3}\NormalTok{), }\StringTok{"}\SpecialCharTok{\textbackslash{}n}\StringTok{"}\NormalTok{)}
\FunctionTok{cat}\NormalTok{(}\StringTok{"效應量大小:"}\NormalTok{, cohen\_d}\SpecialCharTok{$}\NormalTok{magnitude, }\StringTok{"}\SpecialCharTok{\textbackslash{}n}\StringTok{"}\NormalTok{)}
\FunctionTok{cat}\NormalTok{(}\StringTok{"}\SpecialCharTok{\textbackslash{}n}\StringTok{效應量解讀標準:}\SpecialCharTok{\textbackslash{}n}\StringTok{"}\NormalTok{)}
\FunctionTok{cat}\NormalTok{(}\StringTok{"  |d| \textless{} 0.2:微小效應}\SpecialCharTok{\textbackslash{}n}\StringTok{"}\NormalTok{)}
\FunctionTok{cat}\NormalTok{(}\StringTok{"  0.2 \textless{}= |d| \textless{} 0.5:小效應}\SpecialCharTok{\textbackslash{}n}\StringTok{"}\NormalTok{)}
\FunctionTok{cat}\NormalTok{(}\StringTok{"  0.5 \textless{}= |d| \textless{} 0.8:中效應}\SpecialCharTok{\textbackslash{}n}\StringTok{"}\NormalTok{)}
\FunctionTok{cat}\NormalTok{(}\StringTok{"  |d| \textgreater{}= 0.8:大效應}\SpecialCharTok{\textbackslash{}n}\StringTok{"}\NormalTok{)}

\FunctionTok{cat}\NormalTok{(}\StringTok{"}\SpecialCharTok{\textbackslash{}n}\StringTok{"}\NormalTok{)}
\FunctionTok{cat}\NormalTok{(}\StringTok{"{-}"} \SpecialCharTok{\%\textgreater{}\%} \FunctionTok{rep}\NormalTok{(}\DecValTok{60}\NormalTok{) }\SpecialCharTok{\%\textgreater{}\%} \FunctionTok{paste}\NormalTok{(}\AttributeTok{collapse =} \StringTok{""}\NormalTok{), }\StringTok{"}\SpecialCharTok{\textbackslash{}n}\StringTok{"}\NormalTok{)}
\FunctionTok{cat}\NormalTok{(}\StringTok{"五、結論}\SpecialCharTok{\textbackslash{}n}\StringTok{"}\NormalTok{)}
\FunctionTok{cat}\NormalTok{(}\StringTok{"{-}"} \SpecialCharTok{\%\textgreater{}\%} \FunctionTok{rep}\NormalTok{(}\DecValTok{60}\NormalTok{) }\SpecialCharTok{\%\textgreater{}\%} \FunctionTok{paste}\NormalTok{(}\AttributeTok{collapse =} \StringTok{""}\NormalTok{), }\StringTok{"}\SpecialCharTok{\textbackslash{}n}\StringTok{"}\NormalTok{)}
\ControlFlowTok{if}\NormalTok{ (t\_test\_result}\SpecialCharTok{$}\NormalTok{p.value }\SpecialCharTok{\textless{}} \FloatTok{0.05}\NormalTok{) \{}
  \FunctionTok{cat}\NormalTok{(}\StringTok{"統計結論:兩組住院天數有顯著差異 (p \textless{} 0.05)}\SpecialCharTok{\textbackslash{}n}\StringTok{"}\NormalTok{)}
  \FunctionTok{cat}\NormalTok{(}\StringTok{"臨床意義:B 組的平均住院天數比 A 組多約"}\NormalTok{,}
      \FunctionTok{round}\NormalTok{(}\FunctionTok{diff}\NormalTok{(result\_summary}\SpecialCharTok{$}\NormalTok{mean\_los), }\DecValTok{1}\NormalTok{), }\StringTok{"天}\SpecialCharTok{\textbackslash{}n}\StringTok{"}\NormalTok{)}
\NormalTok{\} }\ControlFlowTok{else}\NormalTok{ \{}
  \FunctionTok{cat}\NormalTok{(}\StringTok{"統計結論:兩組住院天數無顯著差異 (p \textgreater{}= 0.05)}\SpecialCharTok{\textbackslash{}n}\StringTok{"}\NormalTok{)}
  \FunctionTok{cat}\NormalTok{(}\StringTok{"臨床意義:目前證據不足以證明兩種治療在住院天數上有差異}\SpecialCharTok{\textbackslash{}n}\StringTok{"}\NormalTok{)}
\NormalTok{\}}

\FunctionTok{cat}\NormalTok{(}\StringTok{"}\SpecialCharTok{\textbackslash{}n}\StringTok{"}\NormalTok{)}
\FunctionTok{cat}\NormalTok{(}\StringTok{"="} \SpecialCharTok{\%\textgreater{}\%} \FunctionTok{rep}\NormalTok{(}\DecValTok{60}\NormalTok{) }\SpecialCharTok{\%\textgreater{}\%} \FunctionTok{paste}\NormalTok{(}\AttributeTok{collapse =} \StringTok{""}\NormalTok{), }\StringTok{"}\SpecialCharTok{\textbackslash{}n}\StringTok{"}\NormalTok{)}
\FunctionTok{cat}\NormalTok{(}\StringTok{"報告結束}\SpecialCharTok{\textbackslash{}n}\StringTok{"}\NormalTok{)}
\FunctionTok{cat}\NormalTok{(}\StringTok{"="} \SpecialCharTok{\%\textgreater{}\%} \FunctionTok{rep}\NormalTok{(}\DecValTok{60}\NormalTok{) }\SpecialCharTok{\%\textgreater{}\%} \FunctionTok{paste}\NormalTok{(}\AttributeTok{collapse =} \StringTok{""}\NormalTok{), }\StringTok{"}\SpecialCharTok{\textbackslash{}n}\StringTok{"}\NormalTok{)}

\FunctionTok{sink}\NormalTok{()  }\CommentTok{\# 關閉 sink,恢復正常輸出}

\CommentTok{\# 確認檔案已產生}
\FunctionTok{cat}\NormalTok{(}\StringTok{"報告已儲存至 report.txt}\SpecialCharTok{\textbackslash{}n}\StringTok{"}\NormalTok{)}
\end{Highlighting}
\end{Shaded}

\begin{verbatim}
報告已儲存至 report.txt
\end{verbatim}

\begin{Shaded}
\begin{Highlighting}[]
\FunctionTok{cat}\NormalTok{(}\StringTok{"檔案大小:"}\NormalTok{, }\FunctionTok{file.info}\NormalTok{(}\StringTok{"report.txt"}\NormalTok{)}\SpecialCharTok{$}\NormalTok{size, }\StringTok{"bytes}\SpecialCharTok{\textbackslash{}n}\StringTok{"}\NormalTok{)}
\end{Highlighting}
\end{Shaded}

\begin{verbatim}
檔案大小: 2162 bytes
\end{verbatim}

\begin{Shaded}
\begin{Highlighting}[]
\CommentTok{\# 顯示報告內容預覽}
\FunctionTok{cat}\NormalTok{(}\StringTok{"}\SpecialCharTok{\textbackslash{}n}\StringTok{{-}{-}{-} 報告內容預覽 {-}{-}{-}}\SpecialCharTok{\textbackslash{}n}\StringTok{"}\NormalTok{)}
\end{Highlighting}
\end{Shaded}

\begin{verbatim}

--- 報告內容預覽 ---
\end{verbatim}

\begin{Shaded}
\begin{Highlighting}[]
\FunctionTok{cat}\NormalTok{(}\FunctionTok{readLines}\NormalTok{(}\StringTok{"report.txt"}\NormalTok{, }\AttributeTok{n =} \DecValTok{20}\NormalTok{), }\AttributeTok{sep =} \StringTok{"}\SpecialCharTok{\textbackslash{}n}\StringTok{"}\NormalTok{)}
\end{Highlighting}
\end{Shaded}

\begin{verbatim}
============================================================ 
        統計分析報告:兩組治療住院天數比較
============================================================ 
報告產生時間: 2025-12-05 20:51:41 

------------------------------------------------------------ 
一、描述性統計
------------------------------------------------------------ 
# A tibble: 2 x 7
  treatment     n mean_los sd_los median_los min_los max_los
  <chr>     <int>    <dbl>  <dbl>      <dbl>   <int>   <int>
1 A            50     4.58   1.07          5       3       7
2 B            50    11.6    2.35         11       8      17

------------------------------------------------------------ 
二、常態性檢定(Shapiro-Wilk Test)
------------------------------------------------------------ 
A 組 p-value: 0.0004 
B 組 p-value: 0.0829 
結論:至少一組資料不符合常態分佈假設
\end{verbatim}

\begin{Shaded}
\begin{Highlighting}[]
\FunctionTok{cat}\NormalTok{(}\StringTok{"...}\SpecialCharTok{\textbackslash{}n}\StringTok{"}\NormalTok{)}
\end{Highlighting}
\end{Shaded}

\begin{verbatim}
...
\end{verbatim}

\section{任務
22:檢查假設}\label{ux4efbux52d9-22ux6aa2ux67e5ux5047ux8a2d}

📋 \textbf{複製這段話,貼給 AI:}

\begin{quote}
剛剛的統計檢定有什麼前提假設?我要怎麼用 R
檢查我的資料有沒有符合這些假設?如果不符合,有什麼替代方法?
\end{quote}

\subsection{假設檢驗}\label{ux5047ux8a2dux6aa2ux9a57}

\begin{Shaded}
\begin{Highlighting}[]
\CommentTok{\# 1. 檢查常態性}
\ControlFlowTok{if}\NormalTok{ (}\SpecialCharTok{!}\FunctionTok{require}\NormalTok{(car, }\AttributeTok{quietly =} \ConstantTok{TRUE}\NormalTok{)) \{}
  \FunctionTok{install.packages}\NormalTok{(}\StringTok{"car"}\NormalTok{, }\AttributeTok{repos =} \StringTok{"https://cloud.r{-}project.org/"}\NormalTok{)}
  \FunctionTok{library}\NormalTok{(car)}
\NormalTok{\}}

\CommentTok{\# Q{-}Q plot}
\FunctionTok{par}\NormalTok{(}\AttributeTok{mfrow =} \FunctionTok{c}\NormalTok{(}\DecValTok{1}\NormalTok{, }\DecValTok{2}\NormalTok{))}
\FunctionTok{qqnorm}\NormalTok{(my\_data}\SpecialCharTok{$}\NormalTok{los[my\_data}\SpecialCharTok{$}\NormalTok{treatment }\SpecialCharTok{==} \StringTok{"A"}\NormalTok{], }\AttributeTok{main =} \StringTok{"Q{-}Q Plot: Treatment A"}\NormalTok{)}
\FunctionTok{qqline}\NormalTok{(my\_data}\SpecialCharTok{$}\NormalTok{los[my\_data}\SpecialCharTok{$}\NormalTok{treatment }\SpecialCharTok{==} \StringTok{"A"}\NormalTok{])}

\FunctionTok{qqnorm}\NormalTok{(my\_data}\SpecialCharTok{$}\NormalTok{los[my\_data}\SpecialCharTok{$}\NormalTok{treatment }\SpecialCharTok{==} \StringTok{"B"}\NormalTok{], }\AttributeTok{main =} \StringTok{"Q{-}Q Plot: Treatment B"}\NormalTok{)}
\FunctionTok{qqline}\NormalTok{(my\_data}\SpecialCharTok{$}\NormalTok{los[my\_data}\SpecialCharTok{$}\NormalTok{treatment }\SpecialCharTok{==} \StringTok{"B"}\NormalTok{])}
\end{Highlighting}
\end{Shaded}

\pandocbounded{\includegraphics[keepaspectratio]{part5_files/figure-pdf/check-assumptions-1.pdf}}

\begin{Shaded}
\begin{Highlighting}[]
\CommentTok{\# 2. 檢查變異數同質性}
\NormalTok{levene\_result }\OtherTok{\textless{}{-}} \FunctionTok{leveneTest}\NormalTok{(los }\SpecialCharTok{\textasciitilde{}}\NormalTok{ treatment, }\AttributeTok{data =}\NormalTok{ my\_data)}
\FunctionTok{print}\NormalTok{(}\StringTok{"Levene\textquotesingle{}s Test for Homogeneity of Variance:"}\NormalTok{)}
\end{Highlighting}
\end{Shaded}

\begin{verbatim}
[1] "Levene's Test for Homogeneity of Variance:"
\end{verbatim}

\begin{Shaded}
\begin{Highlighting}[]
\FunctionTok{print}\NormalTok{(levene\_result)}
\end{Highlighting}
\end{Shaded}

\begin{verbatim}
Levene's Test for Homogeneity of Variance (center = median)
      Df F value    Pr(>F)    
group  1  21.224 0.0000123 ***
      98                      
---
Signif. codes:  0 '***' 0.001 '**' 0.01 '*' 0.05 '.' 0.1 ' ' 1
\end{verbatim}

\begin{Shaded}
\begin{Highlighting}[]
\CommentTok{\# 3. 根據假設檢驗結果選擇適當方法}
\ControlFlowTok{if}\NormalTok{ (shapiro\_a}\SpecialCharTok{$}\NormalTok{p.value }\SpecialCharTok{\textless{}} \FloatTok{0.05} \SpecialCharTok{|}\NormalTok{ shapiro\_b}\SpecialCharTok{$}\NormalTok{p.value }\SpecialCharTok{\textless{}} \FloatTok{0.05}\NormalTok{) \{}
  \FunctionTok{print}\NormalTok{(}\StringTok{"建議:資料不符合常態分佈,應使用 Wilcoxon rank{-}sum test"}\NormalTok{)}
\NormalTok{\} }\ControlFlowTok{else} \ControlFlowTok{if}\NormalTok{ (levene\_result}\SpecialCharTok{$}\StringTok{\textasciigrave{}}\AttributeTok{Pr(\textgreater{}F)}\StringTok{\textasciigrave{}}\NormalTok{[}\DecValTok{1}\NormalTok{] }\SpecialCharTok{\textless{}} \FloatTok{0.05}\NormalTok{) \{}
  \FunctionTok{print}\NormalTok{(}\StringTok{"建議:變異數不同質,應使用 Welch\textquotesingle{}s t{-}test"}\NormalTok{)}
\NormalTok{  welch\_result }\OtherTok{\textless{}{-}} \FunctionTok{t.test}\NormalTok{(los }\SpecialCharTok{\textasciitilde{}}\NormalTok{ treatment, }\AttributeTok{data =}\NormalTok{ my\_data, }\AttributeTok{var.equal =} \ConstantTok{FALSE}\NormalTok{)}
  \FunctionTok{print}\NormalTok{(welch\_result)}
\NormalTok{\} }\ControlFlowTok{else}\NormalTok{ \{}
  \FunctionTok{print}\NormalTok{(}\StringTok{"建議:資料符合所有假設,可使用 Student\textquotesingle{}s t{-}test"}\NormalTok{)}
\NormalTok{\}}
\end{Highlighting}
\end{Shaded}

\begin{verbatim}
[1] "建議:資料不符合常態分佈,應使用 Wilcoxon rank-sum test"
\end{verbatim}

\section{任務
23:類別變數的比較}\label{ux4efbux52d9-23ux985eux5225ux8b8aux6578ux7684ux6bd4ux8f03}

📋 \textbf{複製這段話,貼給 AI:}

\begin{quote}
我想要看 treatment(A vs B)和 gender(M vs F)有沒有關聯。也就是說,A
組和 B 組的男女比例有沒有不同。

請給我適當的統計檢定程式碼,並教我怎麼解讀。
\end{quote}

\subsection{類別變數檢定}\label{ux985eux5225ux8b8aux6578ux6aa2ux5b9a}

\begin{Shaded}
\begin{Highlighting}[]
\CommentTok{\# 建立列聯表}
\NormalTok{contingency\_table }\OtherTok{\textless{}{-}} \FunctionTok{table}\NormalTok{(my\_data}\SpecialCharTok{$}\NormalTok{treatment, my\_data}\SpecialCharTok{$}\NormalTok{gender)}
\FunctionTok{print}\NormalTok{(}\StringTok{"列聯表:"}\NormalTok{)}
\end{Highlighting}
\end{Shaded}

\begin{verbatim}
[1] "列聯表:"
\end{verbatim}

\begin{Shaded}
\begin{Highlighting}[]
\FunctionTok{print}\NormalTok{(contingency\_table)}
\end{Highlighting}
\end{Shaded}

\begin{verbatim}
   
     F  M
  A 23 27
  B 27 23
\end{verbatim}

\begin{Shaded}
\begin{Highlighting}[]
\CommentTok{\# 加上比例}
\NormalTok{prop\_table }\OtherTok{\textless{}{-}} \FunctionTok{prop.table}\NormalTok{(contingency\_table, }\AttributeTok{margin =} \DecValTok{1}\NormalTok{) }\SpecialCharTok{*} \DecValTok{100}
\FunctionTok{print}\NormalTok{(}\StringTok{"各組性別比例(\%):"}\NormalTok{)}
\end{Highlighting}
\end{Shaded}

\begin{verbatim}
[1] "各組性別比例(%):"
\end{verbatim}

\begin{Shaded}
\begin{Highlighting}[]
\FunctionTok{print}\NormalTok{(}\FunctionTok{round}\NormalTok{(prop\_table, }\DecValTok{1}\NormalTok{))}
\end{Highlighting}
\end{Shaded}

\begin{verbatim}
   
     F  M
  A 46 54
  B 54 46
\end{verbatim}

\begin{Shaded}
\begin{Highlighting}[]
\CommentTok{\# 執行卡方檢定}
\NormalTok{chisq\_result }\OtherTok{\textless{}{-}} \FunctionTok{chisq.test}\NormalTok{(contingency\_table)}
\FunctionTok{print}\NormalTok{(}\StringTok{"卡方檢定結果:"}\NormalTok{)}
\end{Highlighting}
\end{Shaded}

\begin{verbatim}
[1] "卡方檢定結果:"
\end{verbatim}

\begin{Shaded}
\begin{Highlighting}[]
\FunctionTok{print}\NormalTok{(chisq\_result)}
\end{Highlighting}
\end{Shaded}

\begin{verbatim}

    Pearson's Chi-squared test with Yates' continuity correction

data:  contingency_table
X-squared = 0.36, df = 1, p-value = 0.5485
\end{verbatim}

\begin{Shaded}
\begin{Highlighting}[]
\CommentTok{\# Fisher\textquotesingle{}s exact test(適用於小樣本)}
\NormalTok{fisher\_result }\OtherTok{\textless{}{-}} \FunctionTok{fisher.test}\NormalTok{(contingency\_table)}
\FunctionTok{print}\NormalTok{(}\StringTok{"Fisher\textquotesingle{}s exact test:"}\NormalTok{)}
\end{Highlighting}
\end{Shaded}

\begin{verbatim}
[1] "Fisher's exact test:"
\end{verbatim}

\begin{Shaded}
\begin{Highlighting}[]
\FunctionTok{print}\NormalTok{(fisher\_result)}
\end{Highlighting}
\end{Shaded}

\begin{verbatim}

    Fisher's Exact Test for Count Data

data:  contingency_table
p-value = 0.5487
alternative hypothesis: true odds ratio is not equal to 1
95 percent confidence interval:
 0.3068457 1.7125734
sample estimates:
odds ratio 
 0.7280058 
\end{verbatim}

\begin{Shaded}
\begin{Highlighting}[]
\CommentTok{\# 解讀結果}
\ControlFlowTok{if}\NormalTok{ (chisq\_result}\SpecialCharTok{$}\NormalTok{p.value }\SpecialCharTok{\textless{}} \FloatTok{0.05}\NormalTok{) \{}
  \FunctionTok{print}\NormalTok{(}\StringTok{"結論:兩組的性別分佈有顯著差異"}\NormalTok{)}
\NormalTok{\} }\ControlFlowTok{else}\NormalTok{ \{}
  \FunctionTok{print}\NormalTok{(}\StringTok{"結論:兩組的性別分佈無顯著差異"}\NormalTok{)}
\NormalTok{\}}
\end{Highlighting}
\end{Shaded}

\begin{verbatim}
[1] "結論:兩組的性別分佈無顯著差異"
\end{verbatim}

\section{任務 24:存活分析 (Survival
Analysis)}\label{ux4efbux52d9-24ux5b58ux6d3bux5206ux6790-survival-analysis}

在臨床研究中,我們經常需要分析「時間到事件」的資料,例如:

\begin{itemize}
\tightlist
\item
  病人從診斷到死亡的時間
\item
  從治療開始到疾病復發的時間
\item
  從手術到再住院的時間
\end{itemize}

這類分析稱為 \textbf{存活分析 (Survival
Analysis)},是臨床研究中非常重要的統計方法。

\subsection{存活分析的特殊之處}\label{ux5b58ux6d3bux5206ux6790ux7684ux7279ux6b8aux4e4bux8655}

存活分析資料有一個特點:\textbf{設限 (Censoring)}

\begin{itemize}
\tightlist
\item
  有些病人在研究結束時還活著(我們不知道他們最終會活多久)
\item
  有些病人中途失聯(lost to follow-up)
\item
  這些「不完整」的觀察值不能直接丟掉,需要特殊處理
\end{itemize}

📋 \textbf{複製這段話,貼給 AI:}

\begin{quote}
我有一個存活分析的資料集
\texttt{patient\_data\_for\_survival.csv},包含:

\begin{itemize}
\tightlist
\item
  patient\_id:病人編號
\item
  treatment:治療組別(Drug\_A vs Drug\_B)
\item
  age:年齡
\item
  gender:性別
\item
  stage:癌症期別(I-IV)
\item
  time:追蹤時間(月)
\item
  status:事件狀態(1=死亡,0=存活/設限)
\end{itemize}

請幫我:

\begin{enumerate}
\def\labelenumi{\arabic{enumi}.}
\tightlist
\item
  用 survival 套件做 Kaplan-Meier 存活曲線
\item
  比較兩個治療組的存活率有沒有差異(log-rank test)
\item
  畫出漂亮的存活曲線圖
\end{enumerate}
\end{quote}

\subsection{讀取存活分析資料}\label{ux8b80ux53d6ux5b58ux6d3bux5206ux6790ux8cc7ux6599}

\begin{Shaded}
\begin{Highlighting}[]
\CommentTok{\# 讀取存活分析資料}
\NormalTok{survival\_data }\OtherTok{\textless{}{-}} \FunctionTok{read.csv}\NormalTok{(}\StringTok{"patient\_data\_for\_survival.csv"}\NormalTok{)}

\CommentTok{\# 檢視資料結構}
\FunctionTok{str}\NormalTok{(survival\_data)}
\end{Highlighting}
\end{Shaded}

\begin{verbatim}
'data.frame':   100 obs. of  7 variables:
 $ patient_id: int  1 2 3 4 5 6 7 8 9 10 ...
 $ treatment : chr  "Drug_A" "Drug_B" "Drug_B" "Drug_B" ...
 $ age       : int  64 45 58 73 64 73 68 50 52 59 ...
 $ gender    : chr  "F" "F" "M" "M" ...
 $ stage     : chr  "III" "III" "II" "I" ...
 $ time      : num  7.8 16.9 17.7 32.2 22 2.4 2.2 36 1.9 12.1 ...
 $ status    : int  1 1 1 1 1 1 1 0 0 1 ...
\end{verbatim}

\begin{Shaded}
\begin{Highlighting}[]
\CommentTok{\# 查看前幾筆}
\FunctionTok{head}\NormalTok{(survival\_data)}
\end{Highlighting}
\end{Shaded}

\begin{verbatim}
  patient_id treatment age gender stage time status
1          1    Drug_A  64      F   III  7.8      1
2          2    Drug_B  45      F   III 16.9      1
3          3    Drug_B  58      M    II 17.7      1
4          4    Drug_B  73      M     I 32.2      1
5          5    Drug_A  64      M    II 22.0      1
6          6    Drug_A  73      F   III  2.4      1
\end{verbatim}

\begin{Shaded}
\begin{Highlighting}[]
\CommentTok{\# 確認事件分佈}
\FunctionTok{table}\NormalTok{(survival\_data}\SpecialCharTok{$}\NormalTok{treatment, survival\_data}\SpecialCharTok{$}\NormalTok{status)}
\end{Highlighting}
\end{Shaded}

\begin{verbatim}
        
          0  1
  Drug_A 16 36
  Drug_B 15 33
\end{verbatim}

\subsection{Kaplan-Meier
存活分析}\label{kaplan-meier-ux5b58ux6d3bux5206ux6790}

\begin{Shaded}
\begin{Highlighting}[]
\CommentTok{\# 安裝並載入必要套件}
\ControlFlowTok{if}\NormalTok{ (}\SpecialCharTok{!}\FunctionTok{require}\NormalTok{(survival, }\AttributeTok{quietly =} \ConstantTok{TRUE}\NormalTok{)) \{}
  \FunctionTok{install.packages}\NormalTok{(}\StringTok{"survival"}\NormalTok{, }\AttributeTok{repos =} \StringTok{"https://cloud.r{-}project.org/"}\NormalTok{)}
\NormalTok{\}}
\ControlFlowTok{if}\NormalTok{ (}\SpecialCharTok{!}\FunctionTok{require}\NormalTok{(ggsurvfit, }\AttributeTok{quietly =} \ConstantTok{TRUE}\NormalTok{)) \{}
  \FunctionTok{install.packages}\NormalTok{(}\StringTok{"ggsurvfit"}\NormalTok{, }\AttributeTok{repos =} \StringTok{"https://cloud.r{-}project.org/"}\NormalTok{)}
\NormalTok{\}}

\FunctionTok{library}\NormalTok{(survival)}
\FunctionTok{library}\NormalTok{(ggsurvfit)}

\CommentTok{\# 建立存活物件}
\CommentTok{\# Surv(time, status) 會建立一個存活時間物件}
\CommentTok{\# time = 追蹤時間, status = 事件是否發生 (1=是, 0=否/設限)}
\NormalTok{surv\_obj }\OtherTok{\textless{}{-}} \FunctionTok{Surv}\NormalTok{(}\AttributeTok{time =}\NormalTok{ survival\_data}\SpecialCharTok{$}\NormalTok{time, }\AttributeTok{event =}\NormalTok{ survival\_data}\SpecialCharTok{$}\NormalTok{status)}

\CommentTok{\# 看看存活物件長什麼樣子}
\FunctionTok{head}\NormalTok{(surv\_obj, }\DecValTok{10}\NormalTok{)}
\end{Highlighting}
\end{Shaded}

\begin{verbatim}
 [1]  7.8  16.9  17.7  32.2  22.0   2.4   2.2  36.0+  1.9+ 12.1 
\end{verbatim}

\begin{Shaded}
\begin{Highlighting}[]
\CommentTok{\# 注意:數字後面的 "+" 表示該觀察值是 censored(設限)}
\end{Highlighting}
\end{Shaded}

\subsection{繪製 Kaplan-Meier
存活曲線}\label{ux7e6aux88fd-kaplan-meier-ux5b58ux6d3bux66f2ux7dda}

\begin{Shaded}
\begin{Highlighting}[]
\CommentTok{\# 使用 ggsurvfit 繪製存活曲線}
\CommentTok{\# survfit2() 是 ggsurvfit 的專用函數,可以更好地追蹤變數名稱}
\FunctionTok{survfit2}\NormalTok{(}\FunctionTok{Surv}\NormalTok{(time, status) }\SpecialCharTok{\textasciitilde{}}\NormalTok{ treatment, }\AttributeTok{data =}\NormalTok{ survival\_data) }\SpecialCharTok{|\textgreater{}}
  \FunctionTok{ggsurvfit}\NormalTok{(}\AttributeTok{linewidth =} \DecValTok{1}\NormalTok{) }\SpecialCharTok{+}
  \FunctionTok{add\_confidence\_interval}\NormalTok{() }\SpecialCharTok{+}
  \FunctionTok{add\_risktable}\NormalTok{() }\SpecialCharTok{+}
  \FunctionTok{add\_pvalue}\NormalTok{(}\AttributeTok{caption =} \StringTok{"Log{-}rank \{p.value\}"}\NormalTok{) }\SpecialCharTok{+}
  \FunctionTok{add\_quantile}\NormalTok{(}\AttributeTok{y\_value =} \FloatTok{0.5}\NormalTok{, }\AttributeTok{linetype =} \StringTok{"dashed"}\NormalTok{, }\AttributeTok{color =} \StringTok{"gray50"}\NormalTok{) }\SpecialCharTok{+}
  \FunctionTok{scale\_color\_manual}\NormalTok{(}\AttributeTok{values =} \FunctionTok{c}\NormalTok{(}\StringTok{"\#E7B800"}\NormalTok{, }\StringTok{"\#2E9FDF"}\NormalTok{)) }\SpecialCharTok{+}
  \FunctionTok{scale\_fill\_manual}\NormalTok{(}\AttributeTok{values =} \FunctionTok{c}\NormalTok{(}\StringTok{"\#E7B800"}\NormalTok{, }\StringTok{"\#2E9FDF"}\NormalTok{)) }\SpecialCharTok{+}
  \FunctionTok{labs}\NormalTok{(}
    \AttributeTok{x =} \StringTok{"追蹤時間(月)"}\NormalTok{,}
    \AttributeTok{y =} \StringTok{"存活機率"}\NormalTok{,}
    \AttributeTok{title =} \StringTok{"Kaplan{-}Meier 存活曲線:Drug A vs Drug B"}
\NormalTok{  ) }\SpecialCharTok{+}
  \FunctionTok{theme\_minimal}\NormalTok{() }\SpecialCharTok{+}
  \FunctionTok{theme}\NormalTok{(}
    \AttributeTok{legend.position =} \StringTok{"bottom"}\NormalTok{,}
    \AttributeTok{plot.title =} \FunctionTok{element\_text}\NormalTok{(}\AttributeTok{hjust =} \FloatTok{0.5}\NormalTok{, }\AttributeTok{face =} \StringTok{"bold"}\NormalTok{)}
\NormalTok{  )}
\end{Highlighting}
\end{Shaded}

\pandocbounded{\includegraphics[keepaspectratio]{part5_files/figure-pdf/survival-km-plot-1.pdf}}

\subsection{查看存活率摘要}\label{ux67e5ux770bux5b58ux6d3bux7387ux6458ux8981}

\begin{Shaded}
\begin{Highlighting}[]
\CommentTok{\# 依照治療組別做 Kaplan{-}Meier 分析}
\NormalTok{km\_fit }\OtherTok{\textless{}{-}} \FunctionTok{survfit}\NormalTok{(}\FunctionTok{Surv}\NormalTok{(time, status) }\SpecialCharTok{\textasciitilde{}}\NormalTok{ treatment, }\AttributeTok{data =}\NormalTok{ survival\_data)}

\CommentTok{\# 查看摘要(6個月、12個月、24個月的存活率)}
\FunctionTok{summary}\NormalTok{(km\_fit, }\AttributeTok{times =} \FunctionTok{c}\NormalTok{(}\DecValTok{6}\NormalTok{, }\DecValTok{12}\NormalTok{, }\DecValTok{24}\NormalTok{))}
\end{Highlighting}
\end{Shaded}

\begin{verbatim}
Call: survfit(formula = Surv(time, status) ~ treatment, data = survival_data)

                treatment=Drug_A 
 time n.risk n.event survival std.err lower 95% CI upper 95% CI
    6     40       9    0.823  0.0536        0.725        0.935
   12     30       9    0.634  0.0690        0.512        0.785
   24     19      11    0.402  0.0709        0.284        0.568

                treatment=Drug_B 
 time n.risk n.event survival std.err lower 95% CI upper 95% CI
    6     34      13    0.724  0.0651        0.607        0.864
   12     27       7    0.575  0.0721        0.450        0.735
   24     19       8    0.405  0.0716        0.286        0.573
\end{verbatim}

\subsection{Log-Rank
Test:比較兩組存活率}\label{log-rank-testux6bd4ux8f03ux5169ux7d44ux5b58ux6d3bux7387}

\begin{Shaded}
\begin{Highlighting}[]
\CommentTok{\# Log{-}rank test(最常用的存活曲線比較方法)}
\NormalTok{logrank\_test }\OtherTok{\textless{}{-}} \FunctionTok{survdiff}\NormalTok{(}\FunctionTok{Surv}\NormalTok{(time, status) }\SpecialCharTok{\textasciitilde{}}\NormalTok{ treatment, }\AttributeTok{data =}\NormalTok{ survival\_data)}
\FunctionTok{print}\NormalTok{(logrank\_test)}
\end{Highlighting}
\end{Shaded}

\begin{verbatim}
Call:
survdiff(formula = Surv(time, status) ~ treatment, data = survival_data)

                  N Observed Expected (O-E)^2/E (O-E)^2/V
treatment=Drug_A 52       36     35.1    0.0219    0.0451
treatment=Drug_B 48       33     33.9    0.0228    0.0451

 Chisq= 0  on 1 degrees of freedom, p= 0.8 
\end{verbatim}

\begin{Shaded}
\begin{Highlighting}[]
\CommentTok{\# 提取 p{-}value}
\NormalTok{logrank\_pvalue }\OtherTok{\textless{}{-}} \DecValTok{1} \SpecialCharTok{{-}} \FunctionTok{pchisq}\NormalTok{(logrank\_test}\SpecialCharTok{$}\NormalTok{chisq, }\FunctionTok{length}\NormalTok{(logrank\_test}\SpecialCharTok{$}\NormalTok{n) }\SpecialCharTok{{-}} \DecValTok{1}\NormalTok{)}
\FunctionTok{print}\NormalTok{(}\FunctionTok{paste}\NormalTok{(}\StringTok{"Log{-}rank test p{-}value:"}\NormalTok{, }\FunctionTok{round}\NormalTok{(logrank\_pvalue, }\DecValTok{4}\NormalTok{)))}
\end{Highlighting}
\end{Shaded}

\begin{verbatim}
[1] "Log-rank test p-value: 0.8319"
\end{verbatim}

\begin{Shaded}
\begin{Highlighting}[]
\CommentTok{\# 解讀結果}
\ControlFlowTok{if}\NormalTok{ (logrank\_pvalue }\SpecialCharTok{\textless{}} \FloatTok{0.05}\NormalTok{) \{}
  \FunctionTok{print}\NormalTok{(}\StringTok{"結論:兩組的存活曲線有顯著差異 (p \textless{} 0.05)"}\NormalTok{)}
\NormalTok{\} }\ControlFlowTok{else}\NormalTok{ \{}
  \FunctionTok{print}\NormalTok{(}\StringTok{"結論:兩組的存活曲線無顯著差異 (p \textgreater{}= 0.05)"}\NormalTok{)}
\NormalTok{\}}
\end{Highlighting}
\end{Shaded}

\begin{verbatim}
[1] "結論:兩組的存活曲線無顯著差異 (p >= 0.05)"
\end{verbatim}

\subsection{計算中位存活時間}\label{ux8a08ux7b97ux4e2dux4f4dux5b58ux6d3bux6642ux9593}

\begin{Shaded}
\begin{Highlighting}[]
\CommentTok{\# 計算各組的中位存活時間}
\FunctionTok{print}\NormalTok{(}\StringTok{"各組中位存活時間(月):"}\NormalTok{)}
\end{Highlighting}
\end{Shaded}

\begin{verbatim}
[1] "各組中位存活時間(月):"
\end{verbatim}

\begin{Shaded}
\begin{Highlighting}[]
\FunctionTok{print}\NormalTok{(km\_fit)}
\end{Highlighting}
\end{Shaded}

\begin{verbatim}
Call: survfit(formula = Surv(time, status) ~ treatment, data = survival_data)

                  n events median 0.95LCL 0.95UCL
treatment=Drug_A 52     36   17.5    12.1    27.9
treatment=Drug_B 48     33   17.7     9.0    32.2
\end{verbatim}

\begin{Shaded}
\begin{Highlighting}[]
\CommentTok{\# 使用 ggsurvfit 的 tidy 方式呈現}
\FunctionTok{survfit2}\NormalTok{(}\FunctionTok{Surv}\NormalTok{(time, status) }\SpecialCharTok{\textasciitilde{}}\NormalTok{ treatment, }\AttributeTok{data =}\NormalTok{ survival\_data) }\SpecialCharTok{|\textgreater{}}
  \FunctionTok{tidy\_survfit}\NormalTok{() }\SpecialCharTok{|\textgreater{}}
  \FunctionTok{head}\NormalTok{(}\DecValTok{10}\NormalTok{)}
\end{Highlighting}
\end{Shaded}

\begin{verbatim}
# A tibble: 10 x 16
    time n.risk n.event n.censor cum.event cum.censor estimate std.error
   <dbl>  <dbl>   <dbl>    <dbl>     <dbl>      <dbl>    <dbl>     <dbl>
 1   0       52       0        0         0          0    1        0     
 2   0.5     52       2        0         2          0    0.962    0.0277
 3   0.7     50       2        0         4          0    0.923    0.0400
 4   1.3     48       0        1         4          1    0.923    0.0400
 5   1.8     47       1        0         5          1    0.903    0.0454
 6   2       46       0        1         5          2    0.903    0.0454
 7   2.4     45       1        0         6          2    0.883    0.0507
 8   2.9     44       1        0         7          2    0.863    0.0557
 9   3.1     43       1        0         8          2    0.843    0.0604
10   5.5     42       1        0         9          2    0.823    0.0651
# i 8 more variables: conf.high <dbl>, conf.low <dbl>, strata <fct>,
#   estimate_type <chr>, estimate_type_label <chr>, monotonicity_type <chr>,
#   strata_label <chr>, conf.level <dbl>
\end{verbatim}

\subsection{進階:Cox
比例風險模型}\label{ux9032ux968ecox-ux6bd4ux4f8bux98a8ux96aaux6a21ux578b}

📋 \textbf{複製這段話,貼給 AI:}

\begin{quote}
我想要做更進階的存活分析,考慮多個變數(治療、年齡、性別、期別)對存活的影響。請幫我用
Cox proportional hazards model 分析,並解釋 Hazard Ratio 的意義。
\end{quote}

\begin{Shaded}
\begin{Highlighting}[]
\CommentTok{\# Cox 比例風險模型}
\CommentTok{\# 可以同時考慮多個變數對存活的影響}
\NormalTok{cox\_model }\OtherTok{\textless{}{-}} \FunctionTok{coxph}\NormalTok{(}
  \FunctionTok{Surv}\NormalTok{(time, status) }\SpecialCharTok{\textasciitilde{}}\NormalTok{ treatment }\SpecialCharTok{+}\NormalTok{ age }\SpecialCharTok{+}\NormalTok{ gender }\SpecialCharTok{+}\NormalTok{ stage,}
  \AttributeTok{data =}\NormalTok{ survival\_data}
\NormalTok{)}

\CommentTok{\# 查看模型結果}
\FunctionTok{summary}\NormalTok{(cox\_model)}
\end{Highlighting}
\end{Shaded}

\begin{verbatim}
Call:
coxph(formula = Surv(time, status) ~ treatment + age + gender + 
    stage, data = survival_data)

  n= 100, number of events= 69 

                    coef exp(coef) se(coef)      z  Pr(>|z|)    
treatmentDrug_B -0.24768   0.78061  0.26698 -0.928  0.353548    
age              0.03947   1.04026  0.01122  3.518  0.000435 ***
genderM         -0.02607   0.97426  0.26378 -0.099  0.921265    
stageII          0.43456   1.54429  0.43274  1.004  0.315280    
stageIII         0.98111   2.66742  0.40840  2.402  0.016291 *  
stageIV          1.77548   5.90313  0.44506  3.989 0.0000663 ***
---
Signif. codes:  0 '***' 0.001 '**' 0.01 '*' 0.05 '.' 0.1 ' ' 1

                exp(coef) exp(-coef) lower .95 upper .95
treatmentDrug_B    0.7806     1.2811    0.4626     1.317
age                1.0403     0.9613    1.0176     1.063
genderM            0.9743     1.0264    0.5810     1.634
stageII            1.5443     0.6475    0.6613     3.606
stageIII           2.6674     0.3749    1.1980     5.939
stageIV            5.9031     0.1694    2.4674    14.123

Concordance= 0.682  (se = 0.029 )
Likelihood ratio test= 29.63  on 6 df,   p=0.00005
Wald test            = 28.58  on 6 df,   p=0.00007
Score (logrank) test = 30.83  on 6 df,   p=0.00003
\end{verbatim}

\begin{Shaded}
\begin{Highlighting}[]
\CommentTok{\# 整理成漂亮的表格}
\ControlFlowTok{if}\NormalTok{ (}\SpecialCharTok{!}\FunctionTok{require}\NormalTok{(gtsummary, }\AttributeTok{quietly =} \ConstantTok{TRUE}\NormalTok{)) \{}
  \FunctionTok{install.packages}\NormalTok{(}\StringTok{"gtsummary"}\NormalTok{, }\AttributeTok{repos =} \StringTok{"https://cloud.r{-}project.org/"}\NormalTok{)}
\NormalTok{\}}
\FunctionTok{library}\NormalTok{(gtsummary)}

\NormalTok{cox\_model }\SpecialCharTok{\%\textgreater{}\%}
  \FunctionTok{tbl\_regression}\NormalTok{(}
    \AttributeTok{exponentiate =} \ConstantTok{TRUE}\NormalTok{,  }\CommentTok{\# 將係數轉換為 Hazard Ratio}
    \AttributeTok{label =} \FunctionTok{list}\NormalTok{(}
\NormalTok{      treatment }\SpecialCharTok{\textasciitilde{}} \StringTok{"治療組別"}\NormalTok{,}
\NormalTok{      age }\SpecialCharTok{\textasciitilde{}} \StringTok{"年齡"}\NormalTok{,}
\NormalTok{      gender }\SpecialCharTok{\textasciitilde{}} \StringTok{"性別"}\NormalTok{,}
\NormalTok{      stage }\SpecialCharTok{\textasciitilde{}} \StringTok{"癌症期別"}
\NormalTok{    )}
\NormalTok{  ) }\SpecialCharTok{\%\textgreater{}\%}
  \FunctionTok{bold\_p}\NormalTok{(}\AttributeTok{t =} \FloatTok{0.05}\NormalTok{) }\SpecialCharTok{\%\textgreater{}\%}
  \FunctionTok{modify\_header}\NormalTok{(}\AttributeTok{label =} \StringTok{"**變數**"}\NormalTok{) }\SpecialCharTok{\%\textgreater{}\%}
  \FunctionTok{modify\_caption}\NormalTok{(}\StringTok{"**Cox 比例風險模型結果**"}\NormalTok{)}
\end{Highlighting}
\end{Shaded}

\begin{table}
\fontsize{12.0pt}{14.0pt}\selectfont
\begin{tabular*}{\linewidth}{@{\extracolsep{\fill}}lccc}
\toprule
\textbf{變數} & \textbf{HR} & \textbf{95\% CI} & \textbf{p-value} \\ 
\midrule\addlinespace[2.5pt]
治療組別 &  &  &  \\ 
    Drug\_A & \textemdash & \textemdash &  \\ 
    Drug\_B & 0.78 & 0.46, 1.32 & 0.4 \\ 
年齡 & 1.04 & 1.02, 1.06 & {\bfseries <0.001} \\ 
性別 &  &  &  \\ 
    F & \textemdash & \textemdash &  \\ 
    M & 0.97 & 0.58, 1.63 & >0.9 \\ 
癌症期別 &  &  &  \\ 
    I & \textemdash & \textemdash &  \\ 
    II & 1.54 & 0.66, 3.61 & 0.3 \\ 
    III & 2.67 & 1.20, 5.94 & {\bfseries 0.016} \\ 
    IV & 5.90 & 2.47, 14.1 & {\bfseries <0.001} \\ 
\bottomrule
\end{tabular*}
\begin{minipage}{\linewidth}
Abbreviations: CI = Confidence Interval, HR = Hazard Ratio\\
\end{minipage}
\end{table}

\subsection{存活分析結果解讀}\label{ux5b58ux6d3bux5206ux6790ux7d50ux679cux89e3ux8b80}

\begin{Shaded}
\begin{Highlighting}[]
\CommentTok{\# 提取 Hazard Ratio 和信賴區間}
\NormalTok{cox\_summary }\OtherTok{\textless{}{-}} \FunctionTok{summary}\NormalTok{(cox\_model)}
\NormalTok{hr }\OtherTok{\textless{}{-}} \FunctionTok{exp}\NormalTok{(cox\_summary}\SpecialCharTok{$}\NormalTok{coefficients[, }\StringTok{"coef"}\NormalTok{])}
\NormalTok{ci\_lower }\OtherTok{\textless{}{-}} \FunctionTok{exp}\NormalTok{(cox\_summary}\SpecialCharTok{$}\NormalTok{coefficients[, }\StringTok{"coef"}\NormalTok{] }\SpecialCharTok{{-}} \FloatTok{1.96} \SpecialCharTok{*}\NormalTok{ cox\_summary}\SpecialCharTok{$}\NormalTok{coefficients[, }\StringTok{"se(coef)"}\NormalTok{])}
\NormalTok{ci\_upper }\OtherTok{\textless{}{-}} \FunctionTok{exp}\NormalTok{(cox\_summary}\SpecialCharTok{$}\NormalTok{coefficients[, }\StringTok{"coef"}\NormalTok{] }\SpecialCharTok{+} \FloatTok{1.96} \SpecialCharTok{*}\NormalTok{ cox\_summary}\SpecialCharTok{$}\NormalTok{coefficients[, }\StringTok{"se(coef)"}\NormalTok{])}

\CommentTok{\# 整理成表格}
\NormalTok{hr\_table }\OtherTok{\textless{}{-}} \FunctionTok{data.frame}\NormalTok{(}
  \AttributeTok{Variable =} \FunctionTok{names}\NormalTok{(hr),}
  \AttributeTok{HR =} \FunctionTok{round}\NormalTok{(hr, }\DecValTok{2}\NormalTok{),}
  \AttributeTok{CI\_Lower =} \FunctionTok{round}\NormalTok{(ci\_lower, }\DecValTok{2}\NormalTok{),}
  \AttributeTok{CI\_Upper =} \FunctionTok{round}\NormalTok{(ci\_upper, }\DecValTok{2}\NormalTok{),}
  \AttributeTok{p\_value =} \FunctionTok{round}\NormalTok{(cox\_summary}\SpecialCharTok{$}\NormalTok{coefficients[, }\StringTok{"Pr(\textgreater{}|z|)"}\NormalTok{], }\DecValTok{4}\NormalTok{)}
\NormalTok{)}
\FunctionTok{print}\NormalTok{(hr\_table)}
\end{Highlighting}
\end{Shaded}

\begin{verbatim}
                       Variable   HR CI_Lower CI_Upper p_value
treatmentDrug_B treatmentDrug_B 0.78     0.46     1.32  0.3535
age                         age 1.04     1.02     1.06  0.0004
genderM                 genderM 0.97     0.58     1.63  0.9213
stageII                 stageII 1.54     0.66     3.61  0.3153
stageIII               stageIII 2.67     1.20     5.94  0.0163
stageIV                 stageIV 5.90     2.47    14.12  0.0001
\end{verbatim}

\begin{Shaded}
\begin{Highlighting}[]
\FunctionTok{cat}\NormalTok{(}\StringTok{"}\SpecialCharTok{\textbackslash{}n}\StringTok{=== Hazard Ratio 解讀指南 ===}\SpecialCharTok{\textbackslash{}n}\StringTok{"}\NormalTok{)}
\end{Highlighting}
\end{Shaded}

\begin{verbatim}

=== Hazard Ratio 解讀指南 ===
\end{verbatim}

\begin{Shaded}
\begin{Highlighting}[]
\FunctionTok{cat}\NormalTok{(}\StringTok{"HR = 1:該因素對存活沒有影響}\SpecialCharTok{\textbackslash{}n}\StringTok{"}\NormalTok{)}
\end{Highlighting}
\end{Shaded}

\begin{verbatim}
HR = 1:該因素對存活沒有影響
\end{verbatim}

\begin{Shaded}
\begin{Highlighting}[]
\FunctionTok{cat}\NormalTok{(}\StringTok{"HR \textgreater{} 1:該因素會增加死亡風險(例如 HR=2 表示風險增加一倍)}\SpecialCharTok{\textbackslash{}n}\StringTok{"}\NormalTok{)}
\end{Highlighting}
\end{Shaded}

\begin{verbatim}
HR > 1:該因素會增加死亡風險(例如 HR=2 表示風險增加一倍)
\end{verbatim}

\begin{Shaded}
\begin{Highlighting}[]
\FunctionTok{cat}\NormalTok{(}\StringTok{"HR \textless{} 1:該因素會降低死亡風險(例如 HR=0.5 表示風險減半)}\SpecialCharTok{\textbackslash{}n}\StringTok{"}\NormalTok{)}
\end{Highlighting}
\end{Shaded}

\begin{verbatim}
HR < 1:該因素會降低死亡風險(例如 HR=0.5 表示風險減半)
\end{verbatim}

\subsection{小結:存活分析的關鍵概念}\label{ux5c0fux7d50ux5b58ux6d3bux5206ux6790ux7684ux95dcux9375ux6982ux5ff5}

\begin{longtable}[]{@{}ll@{}}
\toprule\noalign{}
概念 & 說明 \\
\midrule\noalign{}
\endhead
\bottomrule\noalign{}
\endlastfoot
\textbf{Censoring(設限)} & 觀察結束時事件尚未發生(例如病人還活著) \\
\textbf{Kaplan-Meier} & 估計存活機率隨時間變化的曲線 \\
\textbf{Log-rank test} & 比較兩組或多組存活曲線是否有差異 \\
\textbf{Hazard Ratio (HR)} & 相對風險,HR=2 表示風險是對照組的 2 倍 \\
\textbf{Cox model} & 可同時考慮多個變數對存活的影響 \\
\end{longtable}

\bookmarksetup{startatroot}

\chapter{第六部分:整合與收尾}\label{ux7b2cux516dux90e8ux5206ux6574ux5408ux8207ux6536ux5c3e}

本部分預計時間:20 分鐘

\section{任務
24:完整的分析流程}\label{ux4efbux52d9-24ux5b8cux6574ux7684ux5206ux6790ux6d41ux7a0b}

📋 \textbf{複製這段話,貼給 AI:}

\begin{quote}
請幫我寫一個完整的 R script,從頭到尾做完以下分析:

\begin{enumerate}
\def\labelenumi{\arabic{enumi}.}
\tightlist
\item
  讀取 patient\_data.csv
\item
  用 gtsummary 產出 Table 1(依照 treatment 分組,含 p-value)
\item
  用 ggplot2 畫一個比較兩組 los 的盒狀圖
\item
  做統計檢定比較兩組 los
\item
  把表格存成 Word 檔、圖存成 PNG 檔
\end{enumerate}

請把程式碼整理成一個可以從頭跑到尾的
script,並加上註解說明每一步在做什麼。
\end{quote}

\subsection{完整分析腳本}\label{ux5b8cux6574ux5206ux6790ux8173ux672c}

\begin{Shaded}
\begin{Highlighting}[]
\CommentTok{\# ===========================}
\CommentTok{\# 臨床研究統計分析完整腳本}
\CommentTok{\# ===========================}

\CommentTok{\# 載入必要套件}
\FunctionTok{library}\NormalTok{(gtsummary)}
\FunctionTok{library}\NormalTok{(ggplot2)}
\FunctionTok{library}\NormalTok{(dplyr)}

\CommentTok{\# 安裝並載入 flextable(如果需要)}
\ControlFlowTok{if}\NormalTok{ (}\SpecialCharTok{!}\FunctionTok{require}\NormalTok{(flextable, }\AttributeTok{quietly =} \ConstantTok{TRUE}\NormalTok{)) \{}
  \FunctionTok{install.packages}\NormalTok{(}\StringTok{"flextable"}\NormalTok{, }\AttributeTok{repos =} \StringTok{"https://cloud.r{-}project.org/"}\NormalTok{)}
  \FunctionTok{library}\NormalTok{(flextable)}
\NormalTok{\}}

\CommentTok{\# 1. 讀取資料}
\CommentTok{\# {-}{-}{-}{-}{-}{-}{-}{-}{-}{-}{-}{-}{-}{-}{-}{-}{-}{-}{-}{-}{-}{-}{-}{-}{-}{-}{-}}
\NormalTok{my\_data }\OtherTok{\textless{}{-}} \FunctionTok{read.csv}\NormalTok{(}\StringTok{"patient\_data.csv"}\NormalTok{)}
\FunctionTok{print}\NormalTok{(}\StringTok{"資料讀取成功!"}\NormalTok{)}
\FunctionTok{print}\NormalTok{(}\FunctionTok{paste}\NormalTok{(}\StringTok{"共有"}\NormalTok{, }\FunctionTok{nrow}\NormalTok{(my\_data), }\StringTok{"筆資料"}\NormalTok{))}

\CommentTok{\# 2. 建立 Table 1}
\CommentTok{\# {-}{-}{-}{-}{-}{-}{-}{-}{-}{-}{-}{-}{-}{-}{-}{-}{-}{-}{-}{-}{-}{-}{-}{-}{-}{-}{-}}
\NormalTok{table1 }\OtherTok{\textless{}{-}}\NormalTok{ my\_data }\SpecialCharTok{\%\textgreater{}\%}
  \FunctionTok{select}\NormalTok{(treatment, age, gender, los) }\SpecialCharTok{\%\textgreater{}\%}
  \FunctionTok{tbl\_summary}\NormalTok{(}
    \AttributeTok{by =}\NormalTok{ treatment,}
    \AttributeTok{label =} \FunctionTok{list}\NormalTok{(}
\NormalTok{      age }\SpecialCharTok{\textasciitilde{}} \StringTok{"年齡(歲)"}\NormalTok{,}
\NormalTok{      gender }\SpecialCharTok{\textasciitilde{}} \StringTok{"性別"}\NormalTok{,}
\NormalTok{      los }\SpecialCharTok{\textasciitilde{}} \StringTok{"住院天數(天)"}
\NormalTok{    ),}
    \AttributeTok{statistic =} \FunctionTok{list}\NormalTok{(}
      \FunctionTok{all\_continuous}\NormalTok{() }\SpecialCharTok{\textasciitilde{}} \StringTok{"\{mean\} ± \{sd\}"}\NormalTok{,}
      \FunctionTok{all\_categorical}\NormalTok{() }\SpecialCharTok{\textasciitilde{}} \StringTok{"\{n\} (\{p\}\%)"}
\NormalTok{    )}
\NormalTok{  ) }\SpecialCharTok{\%\textgreater{}\%}
  \FunctionTok{add\_p}\NormalTok{() }\SpecialCharTok{\%\textgreater{}\%}
  \FunctionTok{bold\_p}\NormalTok{(}\AttributeTok{t =} \FloatTok{0.05}\NormalTok{) }\SpecialCharTok{\%\textgreater{}\%}
  \FunctionTok{modify\_header}\NormalTok{(label }\SpecialCharTok{\textasciitilde{}} \StringTok{"**變項**"}\NormalTok{) }\SpecialCharTok{\%\textgreater{}\%}
  \FunctionTok{modify\_spanning\_header}\NormalTok{(}\FunctionTok{c}\NormalTok{(}\StringTok{"stat\_1"}\NormalTok{, }\StringTok{"stat\_2"}\NormalTok{) }\SpecialCharTok{\textasciitilde{}} \StringTok{"**治療組別**"}\NormalTok{)}

\CommentTok{\# 顯示 Table 1}
\FunctionTok{print}\NormalTok{(table1)}

\CommentTok{\# 3. 繪製盒狀圖}
\CommentTok{\# {-}{-}{-}{-}{-}{-}{-}{-}{-}{-}{-}{-}{-}{-}{-}{-}{-}{-}{-}{-}{-}{-}{-}{-}{-}{-}{-}}
\NormalTok{p }\OtherTok{\textless{}{-}} \FunctionTok{ggplot}\NormalTok{(my\_data, }\FunctionTok{aes}\NormalTok{(}\AttributeTok{x =}\NormalTok{ treatment, }\AttributeTok{y =}\NormalTok{ los, }\AttributeTok{fill =}\NormalTok{ treatment)) }\SpecialCharTok{+}
  \FunctionTok{geom\_boxplot}\NormalTok{(}\AttributeTok{alpha =} \FloatTok{0.7}\NormalTok{) }\SpecialCharTok{+}
  \FunctionTok{geom\_jitter}\NormalTok{(}\AttributeTok{width =} \FloatTok{0.2}\NormalTok{, }\AttributeTok{alpha =} \FloatTok{0.5}\NormalTok{, }\AttributeTok{size =} \DecValTok{2}\NormalTok{) }\SpecialCharTok{+}
  \FunctionTok{scale\_fill\_manual}\NormalTok{(}\AttributeTok{values =} \FunctionTok{c}\NormalTok{(}\StringTok{"A"} \OtherTok{=} \StringTok{"steelblue"}\NormalTok{, }\StringTok{"B"} \OtherTok{=} \StringTok{"darkorange"}\NormalTok{)) }\SpecialCharTok{+}
  \FunctionTok{labs}\NormalTok{(}
    \AttributeTok{title =} \StringTok{"兩組治療的住院天數比較"}\NormalTok{,}
    \AttributeTok{x =} \StringTok{"治療組別"}\NormalTok{,}
    \AttributeTok{y =} \StringTok{"住院天數(天)"}
\NormalTok{  ) }\SpecialCharTok{+}
  \FunctionTok{theme\_minimal}\NormalTok{() }\SpecialCharTok{+}
  \FunctionTok{theme}\NormalTok{(}
    \AttributeTok{legend.position =} \StringTok{"none"}\NormalTok{,}
    \AttributeTok{plot.title =} \FunctionTok{element\_text}\NormalTok{(}\AttributeTok{size =} \DecValTok{14}\NormalTok{, }\AttributeTok{face =} \StringTok{"bold"}\NormalTok{, }\AttributeTok{hjust =} \FloatTok{0.5}\NormalTok{)}
\NormalTok{  )}

\CommentTok{\# 顯示圖表}
\FunctionTok{print}\NormalTok{(p)}

\CommentTok{\# 4. 統計檢定}
\CommentTok{\# {-}{-}{-}{-}{-}{-}{-}{-}{-}{-}{-}{-}{-}{-}{-}{-}{-}{-}{-}{-}{-}{-}{-}{-}{-}{-}{-}}
\CommentTok{\# t{-}test}
\NormalTok{t\_test }\OtherTok{\textless{}{-}} \FunctionTok{t.test}\NormalTok{(los }\SpecialCharTok{\textasciitilde{}}\NormalTok{ treatment, }\AttributeTok{data =}\NormalTok{ my\_data)}
\FunctionTok{print}\NormalTok{(}\StringTok{"獨立樣本 t 檢定結果:"}\NormalTok{)}
\FunctionTok{print}\NormalTok{(t\_test)}

\CommentTok{\# Wilcoxon test(無母數替代方法)}
\NormalTok{wilcox\_test }\OtherTok{\textless{}{-}} \FunctionTok{wilcox.test}\NormalTok{(los }\SpecialCharTok{\textasciitilde{}}\NormalTok{ treatment, }\AttributeTok{data =}\NormalTok{ my\_data)}
\FunctionTok{print}\NormalTok{(}\StringTok{"Wilcoxon rank{-}sum 檢定結果:"}\NormalTok{)}
\FunctionTok{print}\NormalTok{(wilcox\_test)}

\CommentTok{\# 5. 儲存輸出}
\CommentTok{\# {-}{-}{-}{-}{-}{-}{-}{-}{-}{-}{-}{-}{-}{-}{-}{-}{-}{-}{-}{-}{-}{-}{-}{-}{-}{-}{-}}
\CommentTok{\# 儲存 Table 1 為 Word 檔}
\NormalTok{table1 }\SpecialCharTok{\%\textgreater{}\%}
  \FunctionTok{as\_flex\_table}\NormalTok{() }\SpecialCharTok{\%\textgreater{}\%}
  \FunctionTok{save\_as\_docx}\NormalTok{(}\AttributeTok{path =} \StringTok{"Table1\_output.docx"}\NormalTok{)}
\FunctionTok{print}\NormalTok{(}\StringTok{"Table 1 已儲存為 Table1\_output.docx"}\NormalTok{)}

\CommentTok{\# 儲存盒狀圖為 PNG 檔}
\FunctionTok{ggsave}\NormalTok{(}
  \AttributeTok{filename =} \StringTok{"boxplot\_los\_comparison.png"}\NormalTok{,}
  \AttributeTok{plot =}\NormalTok{ p,}
  \AttributeTok{width =} \DecValTok{8}\NormalTok{,}
  \AttributeTok{height =} \DecValTok{6}\NormalTok{,}
  \AttributeTok{dpi =} \DecValTok{300}\NormalTok{,}
  \AttributeTok{units =} \StringTok{"in"}
\NormalTok{)}
\FunctionTok{print}\NormalTok{(}\StringTok{"盒狀圖已儲存為 boxplot\_los\_comparison.png"}\NormalTok{)}

\FunctionTok{print}\NormalTok{(}\StringTok{"===== 分析完成!====="}\NormalTok{)}
\end{Highlighting}
\end{Shaded}

\section{任務
25:做成你的範本}\label{ux4efbux52d9-25ux505aux6210ux4f60ux7684ux7bc4ux672c}

📋 \textbf{複製這段話,貼給 AI:}

\begin{quote}
請把剛剛的 script
改成一個「範本」,把檔案名稱、分組變數、要分析的變數都用清楚的變數名稱標示在最上面,這樣我以後只要改最上面的設定,就可以套用到不同的資料。
\end{quote}

\subsection{可重複使用的分析範本}\label{ux53efux91cdux8907ux4f7fux7528ux7684ux5206ux6790ux7bc4ux672c}

\begin{Shaded}
\begin{Highlighting}[]
\CommentTok{\# ========================================}
\CommentTok{\# 臨床研究統計分析範本}
\CommentTok{\# 使用說明:修改下方設定區的參數即可}
\CommentTok{\# ========================================}

\CommentTok{\# ========== 設定區 ==========}
\CommentTok{\# 請根據您的資料修改以下參數}

\CommentTok{\# 檔案設定}
\NormalTok{DATA\_FILE }\OtherTok{\textless{}{-}} \StringTok{"patient\_data.csv"}          \CommentTok{\# 資料檔案名稱}
\NormalTok{OUTPUT\_TABLE }\OtherTok{\textless{}{-}} \StringTok{"Table1\_results.docx"}    \CommentTok{\# 輸出表格檔名}
\NormalTok{OUTPUT\_FIGURE }\OtherTok{\textless{}{-}} \StringTok{"comparison\_plot.png"}   \CommentTok{\# 輸出圖檔名}

\CommentTok{\# 變數設定}
\NormalTok{GROUP\_VAR }\OtherTok{\textless{}{-}} \StringTok{"treatment"}                 \CommentTok{\# 分組變數}
\NormalTok{CONTINUOUS\_VARS }\OtherTok{\textless{}{-}} \FunctionTok{c}\NormalTok{(}\StringTok{"age"}\NormalTok{, }\StringTok{"los"}\NormalTok{)       }\CommentTok{\# 連續變數}
\NormalTok{CATEGORICAL\_VARS }\OtherTok{\textless{}{-}} \FunctionTok{c}\NormalTok{(}\StringTok{"gender"}\NormalTok{)          }\CommentTok{\# 類別變數}
\NormalTok{OUTCOME\_VAR }\OtherTok{\textless{}{-}} \StringTok{"los"}                     \CommentTok{\# 主要結果變數(用於畫圖)}

\CommentTok{\# 標籤設定(中文名稱)}
\NormalTok{LABELS }\OtherTok{\textless{}{-}} \FunctionTok{list}\NormalTok{(}
  \AttributeTok{age =} \StringTok{"年齡(歲)"}\NormalTok{,}
  \AttributeTok{gender =} \StringTok{"性別"}\NormalTok{,}
  \AttributeTok{los =} \StringTok{"住院天數(天)"}\NormalTok{,}
  \AttributeTok{treatment =} \StringTok{"治療組別"}
\NormalTok{)}

\CommentTok{\# 圖表設定}
\NormalTok{PLOT\_TITLE }\OtherTok{\textless{}{-}} \StringTok{"兩組治療的住院天數比較"}
\NormalTok{GROUP\_COLORS }\OtherTok{\textless{}{-}} \FunctionTok{c}\NormalTok{(}\StringTok{"steelblue"}\NormalTok{, }\StringTok{"darkorange"}\NormalTok{)  }\CommentTok{\# 兩組的顏色}

\CommentTok{\# ========== 分析程式(通常不需修改)==========}

\CommentTok{\# 載入套件}
\NormalTok{required\_packages }\OtherTok{\textless{}{-}} \FunctionTok{c}\NormalTok{(}\StringTok{"gtsummary"}\NormalTok{, }\StringTok{"ggplot2"}\NormalTok{, }\StringTok{"dplyr"}\NormalTok{, }\StringTok{"flextable"}\NormalTok{)}

\ControlFlowTok{for}\NormalTok{ (pkg }\ControlFlowTok{in}\NormalTok{ required\_packages) \{}
  \ControlFlowTok{if}\NormalTok{ (}\SpecialCharTok{!}\FunctionTok{require}\NormalTok{(pkg, }\AttributeTok{character.only =} \ConstantTok{TRUE}\NormalTok{, }\AttributeTok{quietly =} \ConstantTok{TRUE}\NormalTok{)) \{}
    \FunctionTok{install.packages}\NormalTok{(pkg, }\AttributeTok{repos =} \StringTok{"https://cloud.r{-}project.org/"}\NormalTok{)}
    \FunctionTok{library}\NormalTok{(pkg, }\AttributeTok{character.only =} \ConstantTok{TRUE}\NormalTok{)}
\NormalTok{  \}}
\NormalTok{\}}

\CommentTok{\# 讀取資料}
\NormalTok{data }\OtherTok{\textless{}{-}} \FunctionTok{read.csv}\NormalTok{(DATA\_FILE)}
\FunctionTok{print}\NormalTok{(}\FunctionTok{paste}\NormalTok{(}\StringTok{"成功讀取"}\NormalTok{, }\FunctionTok{nrow}\NormalTok{(data), }\StringTok{"筆資料"}\NormalTok{))}

\CommentTok{\# 建立變數清單}
\NormalTok{all\_vars }\OtherTok{\textless{}{-}} \FunctionTok{c}\NormalTok{(GROUP\_VAR, CONTINUOUS\_VARS, CATEGORICAL\_VARS)}

\CommentTok{\# 產生 Table 1}
\NormalTok{table\_result }\OtherTok{\textless{}{-}}\NormalTok{ data }\SpecialCharTok{\%\textgreater{}\%}
  \FunctionTok{select}\NormalTok{(}\FunctionTok{all\_of}\NormalTok{(all\_vars)) }\SpecialCharTok{\%\textgreater{}\%}
  \FunctionTok{tbl\_summary}\NormalTok{(}
    \AttributeTok{by =} \FunctionTok{all\_of}\NormalTok{(GROUP\_VAR),}
    \AttributeTok{label =}\NormalTok{ LABELS,}
    \AttributeTok{statistic =} \FunctionTok{list}\NormalTok{(}
      \FunctionTok{all\_continuous}\NormalTok{() }\SpecialCharTok{\textasciitilde{}} \StringTok{"\{mean\} ± \{sd\}"}\NormalTok{,}
      \FunctionTok{all\_categorical}\NormalTok{() }\SpecialCharTok{\textasciitilde{}} \StringTok{"\{n\} (\{p\}\%)"}
\NormalTok{    )}
\NormalTok{  ) }\SpecialCharTok{\%\textgreater{}\%}
  \FunctionTok{add\_p}\NormalTok{() }\SpecialCharTok{\%\textgreater{}\%}
  \FunctionTok{bold\_p}\NormalTok{(}\AttributeTok{t =} \FloatTok{0.05}\NormalTok{)}

\FunctionTok{print}\NormalTok{(table\_result)}

\CommentTok{\# 繪製比較圖}
\NormalTok{plot\_formula }\OtherTok{\textless{}{-}} \FunctionTok{as.formula}\NormalTok{(}\FunctionTok{paste}\NormalTok{(OUTCOME\_VAR, }\StringTok{"\textasciitilde{}"}\NormalTok{, GROUP\_VAR))}

\NormalTok{comparison\_plot }\OtherTok{\textless{}{-}} \FunctionTok{ggplot}\NormalTok{(data, }
                          \FunctionTok{aes\_string}\NormalTok{(}\AttributeTok{x =}\NormalTok{ GROUP\_VAR, }
                                    \AttributeTok{y =}\NormalTok{ OUTCOME\_VAR, }
                                    \AttributeTok{fill =}\NormalTok{ GROUP\_VAR)) }\SpecialCharTok{+}
  \FunctionTok{geom\_boxplot}\NormalTok{(}\AttributeTok{alpha =} \FloatTok{0.7}\NormalTok{) }\SpecialCharTok{+}
  \FunctionTok{geom\_jitter}\NormalTok{(}\AttributeTok{width =} \FloatTok{0.2}\NormalTok{, }\AttributeTok{alpha =} \FloatTok{0.5}\NormalTok{, }\AttributeTok{size =} \DecValTok{2}\NormalTok{) }\SpecialCharTok{+}
  \FunctionTok{scale\_fill\_manual}\NormalTok{(}\AttributeTok{values =}\NormalTok{ GROUP\_COLORS) }\SpecialCharTok{+}
  \FunctionTok{labs}\NormalTok{(}
    \AttributeTok{title =}\NormalTok{ PLOT\_TITLE,}
    \AttributeTok{x =}\NormalTok{ LABELS[[GROUP\_VAR]],}
    \AttributeTok{y =}\NormalTok{ LABELS[[OUTCOME\_VAR]]}
\NormalTok{  ) }\SpecialCharTok{+}
  \FunctionTok{theme\_minimal}\NormalTok{() }\SpecialCharTok{+}
  \FunctionTok{theme}\NormalTok{(}
    \AttributeTok{legend.position =} \StringTok{"none"}\NormalTok{,}
    \AttributeTok{plot.title =} \FunctionTok{element\_text}\NormalTok{(}\AttributeTok{size =} \DecValTok{14}\NormalTok{, }\AttributeTok{face =} \StringTok{"bold"}\NormalTok{, }\AttributeTok{hjust =} \FloatTok{0.5}\NormalTok{)}
\NormalTok{  )}

\FunctionTok{print}\NormalTok{(comparison\_plot)}

\CommentTok{\# 統計檢定}
\NormalTok{test\_result }\OtherTok{\textless{}{-}} \FunctionTok{t.test}\NormalTok{(plot\_formula, }\AttributeTok{data =}\NormalTok{ data)}
\FunctionTok{print}\NormalTok{(test\_result)}

\CommentTok{\# 儲存結果}
\NormalTok{table\_result }\SpecialCharTok{\%\textgreater{}\%}
  \FunctionTok{as\_flex\_table}\NormalTok{() }\SpecialCharTok{\%\textgreater{}\%}
  \FunctionTok{save\_as\_docx}\NormalTok{(}\AttributeTok{path =}\NormalTok{ OUTPUT\_TABLE)}

\FunctionTok{ggsave}\NormalTok{(}
  \AttributeTok{filename =}\NormalTok{ OUTPUT\_FIGURE,}
  \AttributeTok{plot =}\NormalTok{ comparison\_plot,}
  \AttributeTok{width =} \DecValTok{8}\NormalTok{,}
  \AttributeTok{height =} \DecValTok{6}\NormalTok{,}
  \AttributeTok{dpi =} \DecValTok{300}
\NormalTok{)}

\FunctionTok{print}\NormalTok{(}\FunctionTok{paste}\NormalTok{(}\StringTok{"分析完成!結果已儲存至"}\NormalTok{, OUTPUT\_TABLE, }\StringTok{"和"}\NormalTok{, OUTPUT\_FIGURE))}
\end{Highlighting}
\end{Shaded}

\section{任務
26:當你離開這個教室之後}\label{ux4efbux52d9-26ux7576ux4f60ux96e2ux958bux9019ux500bux6559ux5ba4ux4e4bux5f8c}

📋 \textbf{複製這段話,貼給 AI:}

\begin{quote}
我是一個剛學 R 的醫護人員,主要想用 R 做臨床研究的統計分析。請推薦我:

\begin{enumerate}
\def\labelenumi{\arabic{enumi}.}
\tightlist
\item
  3 個最值得學的 R 技能(以我的需求來說)
\item
  2 個適合初學者的免費學習資源
\item
  當我遇到問題時,除了問你之外,還可以去哪裡找答案
\end{enumerate}
\end{quote}

\subsection{持續學習建議}\label{ux6301ux7e8cux5b78ux7fd2ux5efaux8b70}

\subsubsection{最值得學的 3 個 R
技能}\label{ux6700ux503cux5f97ux5b78ux7684-3-ux500b-r-ux6280ux80fd}

\begin{enumerate}
\def\labelenumi{\arabic{enumi}.}
\tightlist
\item
  \textbf{資料整理(dplyr 套件)}

  \begin{itemize}
  \tightlist
  \item
    \texttt{filter()}:篩選資料
  \item
    \texttt{select()}:選擇欄位
  \item
    \texttt{mutate()}:新增計算欄位
  \item
    \texttt{group\_by()} + \texttt{summarise()}:分組統計
  \end{itemize}
\item
  \textbf{統計分析(基礎 R + 專門套件)}

  \begin{itemize}
  \tightlist
  \item
    基礎統計檢定:\texttt{t.test()}, \texttt{wilcox.test()},
    \texttt{chisq.test()}
  \item
    迴歸分析:\texttt{lm()}, \texttt{glm()}
  \item
    存活分析:\texttt{survival} 套件
  \end{itemize}
\item
  \textbf{資料視覺化(ggplot2)}

  \begin{itemize}
  \tightlist
  \item
    基本圖形:散佈圖、長條圖、盒狀圖
  \item
    客製化:主題、顏色、標籤
  \item
    論文品質輸出:高解析度、適當尺寸
  \end{itemize}
\end{enumerate}

\subsubsection{免費學習資源}\label{ux514dux8cbbux5b78ux7fd2ux8cc7ux6e90}

\begin{enumerate}
\def\labelenumi{\arabic{enumi}.}
\item
  \textbf{R for Data Science (線上免費書)}

  \begin{itemize}
  \tightlist
  \item
    網址:https://r4ds.had.co.nz/
  \item
    中文版也有免費線上版本
  \end{itemize}
\item
  \textbf{Swirl 互動式學習套件}

\begin{Shaded}
\begin{Highlighting}[]
\FunctionTok{install.packages}\NormalTok{(}\StringTok{"swirl"}\NormalTok{)}
\FunctionTok{library}\NormalTok{(swirl)}
\FunctionTok{swirl}\NormalTok{()}
\end{Highlighting}
\end{Shaded}
\end{enumerate}

\subsubsection{問題解決資源}\label{ux554fux984cux89e3ux6c7aux8cc7ux6e90}

\begin{enumerate}
\def\labelenumi{\arabic{enumi}.}
\tightlist
\item
  \textbf{Stack Overflow}

  \begin{itemize}
  \tightlist
  \item
    搜尋關鍵字加上 ``R''
  \item
    大部分問題都有人問過
  \end{itemize}
\item
  \textbf{R 官方文件}

  \begin{itemize}
  \tightlist
  \item
    在 R 中輸入 \texttt{?函數名稱} 查看說明
  \item
    例如:\texttt{?t.test}
  \end{itemize}
\item
  \textbf{統計諮詢}

  \begin{itemize}
  \tightlist
  \item
    醫院或學校的統計諮詢中心
  \item
    線上統計論壇(如 Cross Validated)
  \end{itemize}
\end{enumerate}

\subsection{練習建議}\label{ux7df4ux7fd2ux5efaux8b70}

\begin{longtable}[]{@{}ccc@{}}
\toprule\noalign{}
週次 & 練習重點 & 建議時間 \\
\midrule\noalign{}
\endhead
\bottomrule\noalign{}
\endlastfoot
1 & 重複今天的內容,用自己的資料 & 每天 30 分鐘 \\
2 & 學習 dplyr 資料整理 & 每天 30 分鐘 \\
3 & 嘗試不同的統計檢定 & 每週 2-3 次 \\
4 & 製作更複雜的圖表 & 每週 2-3 次 \\
\end{longtable}

\section{今天學會的超能力總結}\label{ux4ecaux5929ux5b78ux6703ux7684ux8d85ux80fdux529bux7e3dux7d50}

\begin{itemize}
\tightlist
\item
  ✅ 把問題描述清楚,讓 AI 幫你寫程式
\item
  ✅ 看懂 AI 給的程式碼大概在做什麼
\item
  ✅ 當程式出錯,知道怎麼問 AI 修正
\item
  ✅ 產出可以放進論文的表格和圖表
\item
  ✅ 有一個可以重複使用的分析範本
\end{itemize}

\textbf{你不需要記住任何語法,但要記住如何問對問題!}

\bookmarksetup{startatroot}

\chapter{附錄:常見技術問題與解決方案}\label{ux9644ux9304ux5e38ux898bux6280ux8853ux554fux984cux8207ux89e3ux6c7aux65b9ux6848}

\subsection{R 環境設置}\label{r-ux74b0ux5883ux8a2dux7f6e}

\subsubsection{套件安裝問題}\label{ux5957ux4ef6ux5b89ux88ddux554fux984c}

\begin{Shaded}
\begin{Highlighting}[]
\CommentTok{\# 如果安裝失敗,可以嘗試選擇不同的 CRAN mirror}
\FunctionTok{options}\NormalTok{(}\AttributeTok{repos =} \FunctionTok{c}\NormalTok{(}\AttributeTok{CRAN =} \StringTok{"https://cloud.r{-}project.org/"}\NormalTok{))}
\FunctionTok{install.packages}\NormalTok{(}\FunctionTok{c}\NormalTok{(}\StringTok{"ggplot2"}\NormalTok{, }\StringTok{"gtsummary"}\NormalTok{, }\StringTok{"dplyr"}\NormalTok{))}
\end{Highlighting}
\end{Shaded}

\subsubsection{中文編碼問題}\label{ux4e2dux6587ux7de8ux78bcux554fux984c}

\begin{Shaded}
\begin{Highlighting}[]
\CommentTok{\# 設定正確的編碼}
\FunctionTok{Sys.setlocale}\NormalTok{(}\StringTok{"LC\_ALL"}\NormalTok{, }\StringTok{"Chinese"}\NormalTok{)}
\end{Highlighting}
\end{Shaded}

\subsection{資料讀取問題}\label{ux8cc7ux6599ux8b80ux53d6ux554fux984c}

\subsubsection{CSV 檔案路徑}\label{csv-ux6a94ux6848ux8defux5f91}

\begin{Shaded}
\begin{Highlighting}[]
\CommentTok{\# 檢查工作目錄}
\FunctionTok{getwd}\NormalTok{()}

\CommentTok{\# 設定工作目錄到資料檔案所在位置}
\FunctionTok{setwd}\NormalTok{(}\StringTok{"/path/to/your/data"}\NormalTok{)}

\CommentTok{\# 或使用絕對路徑}
\NormalTok{my\_data }\OtherTok{\textless{}{-}} \FunctionTok{read.csv}\NormalTok{(}\StringTok{"/完整路徑/patient\_data.csv"}\NormalTok{)}
\end{Highlighting}
\end{Shaded}

\subsubsection{編碼問題}\label{ux7de8ux78bcux554fux984c}

\begin{Shaded}
\begin{Highlighting}[]
\CommentTok{\# 如果中文顯示亂碼}
\NormalTok{my\_data }\OtherTok{\textless{}{-}} \FunctionTok{read.csv}\NormalTok{(}\StringTok{"patient\_data.csv"}\NormalTok{, }\AttributeTok{fileEncoding =} \StringTok{"UTF{-}8"}\NormalTok{)}
\end{Highlighting}
\end{Shaded}

\section{教學節奏建議}\label{ux6559ux5b78ux7bc0ux594fux5efaux8b70}

\subsection{第一部分(40分鐘)}\label{ux7b2cux4e00ux90e8ux520640ux5206ux9418}

\begin{itemize}
\tightlist
\item
  任務 1-2:15 分鐘(重點是建立信心)
\item
  任務 3:5 分鐘(如果沒錯誤可跳過)
\item
  任務 4:10 分鐘(概念理解)
\item
  任務 5:10 分鐘(實際安裝)
\end{itemize}

\subsection{第二部分(20分鐘)}\label{ux7b2cux4e8cux90e8ux520620ux5206ux9418}

\begin{itemize}
\tightlist
\item
  任務 6:8 分鐘(重點是成功讀取)
\item
  任務 7:7 分鐘(資料探索)
\item
  任務 8:5 分鐘(理解資料)
\end{itemize}

\subsection{第三部分(40分鐘)}\label{ux7b2cux4e09ux90e8ux520640ux5206ux9418}

\begin{itemize}
\tightlist
\item
  任務 9:5 分鐘(概念建立)
\item
  任務 10:10 分鐘(第一個表格)
\item
  任務 11:8 分鐘(加上統計)
\item
  任務 12:7 分鐘(理解方法選擇)
\item
  任務 13:10 分鐘(客製化)
\end{itemize}

\subsection{第四部分(40分鐘)}\label{ux7b2cux56dbux90e8ux520640ux5206ux9418}

\begin{itemize}
\tightlist
\item
  任務 15:8 分鐘(基本圖)
\item
  任務 16:12 分鐘(美化)
\item
  任務 17:10 分鐘(加資料點)
\item
  任務 18:10 分鐘(其他圖型,讓學員選擇)
\end{itemize}

\subsection{第五部分(30分鐘)}\label{ux7b2cux4e94ux90e8ux520630ux5206ux9418}

\begin{itemize}
\tightlist
\item
  任務 20:12 分鐘(統計方法選擇)
\item
  任務 21:8 分鐘(結果解讀)
\item
  任務 22:10 分鐘(假設檢驗)
\end{itemize}

\subsection{第六部分(20分鐘)}\label{ux7b2cux516dux90e8ux520620ux5206ux9418}

\begin{itemize}
\tightlist
\item
  任務 24:10 分鐘(完整腳本)
\item
  任務 25:10 分鐘(範本化)
\end{itemize}

\section{應急方案}\label{ux61c9ux6025ux65b9ux6848}

\subsection{如果 AI
無法使用}\label{ux5982ux679c-ai-ux7121ux6cd5ux4f7fux7528}

準備預寫好的程式碼片段,但仍要強調: - 這不是背誦的重點 -
重點是理解如何描述問題 - AI 恢復後繼續用 AI 輔助方式

\section{評估學習成效}\label{ux8a55ux4f30ux5b78ux7fd2ux6210ux6548}

\subsection{課程結束前檢查清單}\label{ux8ab2ux7a0bux7d50ux675fux524dux6aa2ux67e5ux6e05ux55ae}

學員應該能夠: - {[} {]} 成功執行 AI 提供的 R 程式碼 - {[} {]}
產出一個基本的描述性統計表格 - {[} {]} 畫出一個基本的比較圖表 - {[} {]}
知道在遇到錯誤時如何求助(描述問題給 AI) - {[} {]} 理解 R
程式碼的基本結構(變數、函數、參數)

\section{2025 年臨床研究者的 R
語言最佳套件組合指南}\label{ux5e74ux81e8ux5e8aux7814ux7a76ux8005ux7684-r-ux8a9eux8a00ux6700ux4f73ux5957ux4ef6ux7d44ux5408ux6307ux5357}

\textbf{Quarto 取代 R Markdown 成為新標準,gtsummary + flextable
主導臨床表格製作,而 ggsurvfit 正式取代 survminer
成為存活分析視覺化首選。} 本報告基於 Reddit r/rstats、GitHub
活躍度、CRAN 更新紀錄及 Posit 官方文件,為臨床醫學研究者彙整 2025 年 R
套件生態系統的最新發展與最佳實踐。

\section{資料視覺化:ggplot2
生態系的黃金組合}\label{ux8cc7ux6599ux8996ux89baux5316ggplot2-ux751fux614bux7cfbux7684ux9ec3ux91d1ux7d44ux5408}

臨床研究視覺化的核心仍是
\textbf{ggplot2},但其擴展套件生態系已趨成熟。2025
年社群最推薦的組合是:

\textbf{基礎組合:ggplot2 + ggpubr + ggsci + patchwork}

\begin{longtable}[]{@{}llll@{}}
\toprule\noalign{}
套件 & GitHub Stars & 最後更新 & 核心功能 \\
\midrule\noalign{}
\endhead
\bottomrule\noalign{}
\endlastfoot
\textbf{patchwork} & 2,600+ & 2024.09 & 圖表組合(取代 gridExtra) \\
\textbf{ggpubr} & 活躍維護 & 2024 & 出版級統計圖表 + 顯著性標註 \\
\textbf{ggsci} & 活躍維護 & 2024 & 醫學期刊配色(NEJM、Lancet、JAMA) \\
\textbf{ggrepel} & 1,200+ & 2024.11 & 避免標籤重疊 \\
\textbf{ggstatsplot} & 2,100+ & 2025.08 & 自動統計分析與視覺化整合 \\
\end{longtable}

\textbf{patchwork}
已成為圖表組合的事實標準,語法直覺:\texttt{p1\ +\ p2}
水平排列、\texttt{p1\ /\ p2} 垂直堆疊。\textbf{ggsci} 提供
\texttt{scale\_color\_nejm()}、\texttt{scale\_fill\_lancet()}
等函數,讓圖表配色直接符合期刊風格。

\subsection{臨床專用圖表的最佳選擇}\label{ux81e8ux5e8aux5c08ux7528ux5716ux8868ux7684ux6700ux4f73ux9078ux64c7}

\textbf{存活曲線:ggsurvfit(2025 標準)}

survminer 已逐漸被 \textbf{ggsurvfit} 取代。關鍵差異在於 ggsurvfit
回傳真正的 ggplot 物件,可完全客製化:

\begin{Shaded}
\begin{Highlighting}[]
\FunctionTok{library}\NormalTok{(ggsurvfit)}
\FunctionTok{survfit2}\NormalTok{(}\FunctionTok{Surv}\NormalTok{(time, status) }\SpecialCharTok{\textasciitilde{}}\NormalTok{ group, }\AttributeTok{data =}\NormalTok{ df) }\SpecialCharTok{|\textgreater{}}
  \FunctionTok{ggsurvfit}\NormalTok{() }\SpecialCharTok{+}
  \FunctionTok{add\_risktable}\NormalTok{() }\SpecialCharTok{+}
  \FunctionTok{add\_confidence\_interval}\NormalTok{() }\SpecialCharTok{+}
  \FunctionTok{scale\_color\_nejm}\NormalTok{()}
\end{Highlighting}
\end{Shaded}

ggsurvfit 由 pharmaverse 團隊開發,原生支援 CDISC ADaM
資料格式,風險表會自動對齊縮放------這是 survminer 長期被詬病的問題。

\begin{itemize}
\tightlist
\item
  \textbf{ggsurvfit}: https://cran.r-project.org/package=ggsurvfit(2024
  活躍更新)
\item
  \textbf{survminer}: 仍可用但開發較不活躍,建議新專案使用 ggsurvfit
\end{itemize}

\textbf{森林圖:forestploter(2025 首選)}

\textbf{forestploter}
將森林圖視為對齊的表格處理,提供最大的版面控制彈性。相較於 forestplot
套件的 grid 繪圖系統,forestploter 更容易產出期刊品質的圖表:

\begin{itemize}
\tightlist
\item
  \textbf{forestploter}:
  https://cran.r-project.org/package=forestploter(2024 活躍)
\item
  \textbf{meta::forest()}: 適合標準 meta-analysis,內建 JAMA、RevMan5
  版型
\item
  \textbf{ggforestplot}: 適合生物標記研究的垂直排列版型
\end{itemize}

\subsection{互動式圖表比較}\label{ux4e92ux52d5ux5f0fux5716ux8868ux6bd4ux8f03}

\begin{longtable}[]{@{}llll@{}}
\toprule\noalign{}
套件 & 基礎技術 & 最佳用途 & 授權 \\
\midrule\noalign{}
\endhead
\bottomrule\noalign{}
\endlastfoot
\textbf{plotly} & Plotly.js & ggplot2 快速轉互動(ggplotly) & 開源 \\
\textbf{ggiraph} & htmlwidgets & 工具提示 + 連動圖表 & 開源 \\
\textbf{echarts4r} & Apache ECharts & Shiny 儀表板 & 開源 \\
\textbf{highcharter} & Highcharts & 商業級品質 & 商用需授權 \\
\end{longtable}

\textbf{plotly} 仍是最簡便的選擇------現有 ggplot2 程式碼只需包裹
\texttt{ggplotly()}
即可互動化。需要進階工具提示或圖表連動時,\textbf{ggiraph}
是更好的選擇。

\section{流程圖製作:CONSORT/STROBE
圖的最佳解決方案}\label{ux6d41ux7a0bux5716ux88fdux4f5cconsortstrobe-ux5716ux7684ux6700ux4f73ux89e3ux6c7aux65b9ux6848}

\subsection{2025 年推薦:flowchart
套件}\label{ux5e74ux63a8ux85a6flowchart-ux5957ux4ef6}

\textbf{flowchart} 套件(2024 年 2 月首發,2024 年 11 月更新至
v0.6.0)採用 tidyverse 工作流程,可從資料框自動計算受試者人數:

\begin{Shaded}
\begin{Highlighting}[]
\FunctionTok{library}\NormalTok{(flowchart)}
\NormalTok{data }\SpecialCharTok{|\textgreater{}}
  \FunctionTok{as\_fc}\NormalTok{(}\AttributeTok{label =} \StringTok{"Screened"}\NormalTok{) }\SpecialCharTok{|\textgreater{}}
  \FunctionTok{fc\_filter}\NormalTok{(eligible }\SpecialCharTok{==} \ConstantTok{TRUE}\NormalTok{, }\AttributeTok{label =} \StringTok{"Eligible"}\NormalTok{) }\SpecialCharTok{|\textgreater{}}
  \FunctionTok{fc\_split}\NormalTok{(group, }\AttributeTok{label =} \StringTok{"Randomized"}\NormalTok{) }\SpecialCharTok{|\textgreater{}}
  \FunctionTok{fc\_draw}\NormalTok{()}
\end{Highlighting}
\end{Shaded}

\textbf{優勢}:數據變更時圖表自動更新,完全可重現。

\begin{longtable}[]{@{}lllll@{}}
\toprule\noalign{}
套件 & 用途 & CRAN & 最後更新 & 學習曲線 \\
\midrule\noalign{}
\endhead
\bottomrule\noalign{}
\endlastfoot
\textbf{flowchart} & CONSORT/STROBE & ✅ & 2024.11 & 低(tidyverse) \\
\textbf{consort} & CONSORT 專用 & ✅ & 2024 & 低 \\
\textbf{DiagrammeR} & 通用流程圖 & ✅ & 2024.02 & 中高 \\
\textbf{ggflowchart} & 簡單流程圖 & ✅ & 2023.05 & 低 \\
\end{longtable}

\textbf{consort} 套件(https://github.com/adayim/consort)專為 CONSORT
聲明設計,適合需要嚴格遵循 CONSORT 格式的研究。\textbf{DiagrammeR}
則適合需要完全客製化的複雜流程圖,使用 Graphviz DOT 語言。

\section{表格製作:gtsummary + flextable
是黃金組合}\label{ux8868ux683cux88fdux4f5cgtsummary-flextable-ux662fux9ec3ux91d1ux7d44ux5408}

\subsection{Table
1(基線特徵表)}\label{table-1ux57faux7ddaux7279ux5fb5ux8868}

\textbf{gtsummary} 已成為臨床研究 Table 1 的事實標準,取代了較舊的
tableone:

\begin{longtable}[]{@{}lllll@{}}
\toprule\noalign{}
功能 & gtsummary & tableone & table1 & arsenal \\
\midrule\noalign{}
\endhead
\bottomrule\noalign{}
\endlastfoot
臨床研究導向 & ⭐⭐⭐ & ⭐⭐⭐ & ⭐⭐ & ⭐⭐⭐ \\
自動變數偵測 & ✅ & ✅ & ❌ & ✅ \\
統計檢定 p 值 & ✅ add\_p() & ✅ & 有限 & ✅ \\
SMD 支援 & ✅ add\_difference() & ⭐⭐⭐ & ❌ & ✅ \\
Word 輸出 & ⭐⭐⭐ & ⭐⭐ & ❌ & ⭐⭐ \\
活躍開發 & ⭐⭐⭐ & ⭐⭐ & ⭐⭐ & ⭐⭐ \\
\end{longtable}

\textbf{gtsummary} 最後更新:2024 年 11 月 30 日(v2.4.0),由 Memorial
Sloan Kettering 的 Daniel Sjoberg 開發維護。

\begin{Shaded}
\begin{Highlighting}[]
\FunctionTok{library}\NormalTok{(gtsummary)}
\NormalTok{data }\SpecialCharTok{|\textgreater{}}
  \FunctionTok{tbl\_summary}\NormalTok{(}
    \AttributeTok{by =}\NormalTok{ treatment,}
    \AttributeTok{include =} \FunctionTok{c}\NormalTok{(age, sex, bmi),}
    \AttributeTok{statistic =} \FunctionTok{list}\NormalTok{(}\FunctionTok{all\_continuous}\NormalTok{() }\SpecialCharTok{\textasciitilde{}} \StringTok{"\{mean\} (\{sd\})"}\NormalTok{)}
\NormalTok{  ) }\SpecialCharTok{|\textgreater{}}
  \FunctionTok{add\_p}\NormalTok{() }\SpecialCharTok{|\textgreater{}}
  \FunctionTok{add\_overall}\NormalTok{() }\SpecialCharTok{|\textgreater{}}
  \FunctionTok{bold\_labels}\NormalTok{()}
\end{Highlighting}
\end{Shaded}

\subsection{輸出格式化}\label{ux8f38ux51faux683cux5f0fux5316}

\textbf{Word 輸出(期刊投稿必備)}:使用 \textbf{flextable}

\begin{Shaded}
\begin{Highlighting}[]
\FunctionTok{tbl\_summary}\NormalTok{(data, }\AttributeTok{by =}\NormalTok{ group) }\SpecialCharTok{|\textgreater{}}
  \FunctionTok{as\_flex\_table}\NormalTok{() }\SpecialCharTok{|\textgreater{}}
\NormalTok{  flextable}\SpecialCharTok{::}\FunctionTok{save\_as\_docx}\NormalTok{(}\AttributeTok{path =} \StringTok{"Table1.docx"}\NormalTok{)}
\end{Highlighting}
\end{Shaded}

\begin{longtable}[]{@{}lllll@{}}
\toprule\noalign{}
套件 & Word & HTML & PDF & 推薦場景 \\
\midrule\noalign{}
\endhead
\bottomrule\noalign{}
\endlastfoot
\textbf{flextable} & ⭐⭐⭐ & ⭐⭐ & ⭐⭐ & 期刊投稿(Word) \\
\textbf{gt} & ⭐⭐ & ⭐⭐⭐ & ⭐⭐ & 網頁報告 \\
\textbf{kableExtra} & ⭐ & ⭐⭐⭐ & ⭐⭐⭐ & LaTeX/PDF \\
\textbf{huxtable} & ⭐⭐ & ⭐⭐ & ⭐⭐ & 多格式輸出 \\
\end{longtable}

\textbf{gt} 由 Posit 開發,HTML 輸出最美觀,但 Word 支援仍不如 flextable
穩定。

\section{環境與套件管理:2025
年標準工具鏈}\label{ux74b0ux5883ux8207ux5957ux4ef6ux7ba1ux74062025-ux5e74ux6a19ux6e96ux5de5ux5177ux93c8}

\subsection{專案環境管理:renv(packrat
已棄用)}\label{ux5c08ux6848ux74b0ux5883ux7ba1ux7406renvpackrat-ux5df2ux68c4ux7528}

\textbf{packrat 已被官方軟棄用},renv 是唯一推薦選擇:

\begin{Shaded}
\begin{Highlighting}[]
\NormalTok{renv}\SpecialCharTok{::}\FunctionTok{init}\NormalTok{()        }\CommentTok{\# 初始化專案}
\NormalTok{renv}\SpecialCharTok{::}\FunctionTok{snapshot}\NormalTok{()    }\CommentTok{\# 儲存套件版本}
\NormalTok{renv}\SpecialCharTok{::}\FunctionTok{restore}\NormalTok{()     }\CommentTok{\# 還原環境}
\end{Highlighting}
\end{Shaded}

\begin{itemize}
\tightlist
\item
  \textbf{renv}: https://cran.r-project.org/package=renv(2024.10 更新)
\item
  使用 JSON 格式的 \texttt{renv.lock} 記錄所有套件版本
\item
  全域快取節省磁碟空間
\end{itemize}

\subsection{套件安裝:pak 優於
install.packages}\label{ux5957ux4ef6ux5b89ux88ddpak-ux512aux65bc-install.packages}

\textbf{pak} 提供平行下載、相依性解析、多來源支援:

\begin{Shaded}
\begin{Highlighting}[]
\NormalTok{pak}\SpecialCharTok{::}\FunctionTok{pak}\NormalTok{(}\StringTok{"tidyverse"}\NormalTok{)              }\CommentTok{\# CRAN}
\NormalTok{pak}\SpecialCharTok{::}\FunctionTok{pak}\NormalTok{(}\StringTok{"tidyverse/dplyr"}\NormalTok{)        }\CommentTok{\# GitHub}
\NormalTok{pak}\SpecialCharTok{::}\FunctionTok{pak}\NormalTok{(}\StringTok{"bioc::Biobase"}\NormalTok{)          }\CommentTok{\# Bioconductor}
\end{Highlighting}
\end{Shaded}

\begin{longtable}[]{@{}lll@{}}
\toprule\noalign{}
功能 & pak & install.packages() \\
\midrule\noalign{}
\endhead
\bottomrule\noalign{}
\endlastfoot
平行下載 & ✅ & ❌ \\
相依性衝突檢查 & ✅ & ❌ \\
GitHub/Bioconductor & ✅ 原生支援 & ❌ 需額外套件 \\
系統相依性提示 & ✅ & ❌ \\
\end{longtable}

\subsection{R 版本管理:rig}\label{r-ux7248ux672cux7ba1ux7406rig}

\textbf{rig}(https://github.com/r-lib/rig)是跨平台的 R 版本管理工具:

\begin{Shaded}
\begin{Highlighting}[]
\ExtensionTok{rig}\NormalTok{ add release }\CommentTok{\# 安裝最新穩定版}
\ExtensionTok{rig}\NormalTok{ add 4.3.2   }\CommentTok{\# 安裝特定版本}
\ExtensionTok{rig}\NormalTok{ default 4.4 }\CommentTok{\# 設定預設版本}
\ExtensionTok{rig}\NormalTok{ rstudio 4.3 }\CommentTok{\# 以特定版本啟動 RStudio / Antigravity}
\end{Highlighting}
\end{Shaded}

macOS 使用者也可選擇 \textbf{RSwitch} 作為 GUI 替代方案。

\section{現代 R
語法:管道運算子與資料處理}\label{ux73feux4ee3-r-ux8a9eux6cd5ux7ba1ux9053ux904bux7b97ux5b50ux8207ux8cc7ux6599ux8655ux7406}

\subsection{管道運算子:native \textbar\textgreater{}
已成主流}\label{ux7ba1ux9053ux904bux7b97ux5b50native-ux5df2ux6210ux4e3bux6d41}

\textbf{2025 社群共識}:新程式碼優先使用 native pipe
\texttt{\textbar{}\textgreater{}},magrittr \texttt{\%\textgreater{}\%}
保留給需要進階 placeholder 功能時使用。

\begin{longtable}[]{@{}lll@{}}
\toprule\noalign{}
特性 & Native \texttt{\textbar{}\textgreater{}} & Magrittr
\texttt{\%\textgreater{}\%} \\
\midrule\noalign{}
\endhead
\bottomrule\noalign{}
\endlastfoot
效能 & 零額外開銷 & 微小開銷 \\
Placeholder & \texttt{\_}(僅限具名參數) & \texttt{.}(彈性) \\
相依性 & 無(R 4.1+) & 需安裝 magrittr \\
多次使用 placeholder & ❌ & ✅ \\
\end{longtable}

\begin{Shaded}
\begin{Highlighting}[]
\CommentTok{\# Native pipe(推薦)}
\NormalTok{df }\SpecialCharTok{|\textgreater{}} \FunctionTok{filter}\NormalTok{(age }\SpecialCharTok{\textgreater{}} \DecValTok{18}\NormalTok{) }\SpecialCharTok{|\textgreater{}} \FunctionTok{summarise}\NormalTok{(}\AttributeTok{mean\_bmi =} \FunctionTok{mean}\NormalTok{(bmi))}

\CommentTok{\# 需要 magrittr 的情況}
\NormalTok{df }\SpecialCharTok{\%\textgreater{}\%}\NormalTok{ \{}\FunctionTok{paste}\NormalTok{(.}\SpecialCharTok{$}\NormalTok{name, .}\SpecialCharTok{$}\NormalTok{id)\}  }\CommentTok{\# placeholder 多次使用}
\end{Highlighting}
\end{Shaded}

\subsection{data.table vs dplyr vs
polars}\label{data.table-vs-dplyr-vs-polars}

\begin{longtable}[]{@{}ll@{}}
\toprule\noalign{}
資料規模 & 推薦工具 \\
\midrule\noalign{}
\endhead
\bottomrule\noalign{}
\endlastfoot
\textless{} 100 萬列 & dplyr \\
100 萬 - 1000 萬列 & dtplyr 或 data.table \\
\textgreater{} 1000 萬列或記憶體不足 & data.table 或 polars \\
\end{longtable}

\textbf{polars}(https://rpolars.github.io)正在 R
社群快速成長,效能可達 dplyr 的 \textbf{15-25 倍},但尚未上架
CRAN。\textbf{dtplyr} 讓你用 dplyr 語法獲得 data.table 效能。

\subsection{Tidyverse
核心套件評估}\label{tidyverse-ux6838ux5fc3ux5957ux4ef6ux8a55ux4f30}

\begin{longtable}[]{@{}lll@{}}
\toprule\noalign{}
套件 & 狀態 & 說明 \\
\midrule\noalign{}
\endhead
\bottomrule\noalign{}
\endlastfoot
ggplot2 & ⭐ 必備 & 視覺化唯一標準 \\
dplyr & ⭐ 必備 & 資料處理核心 \\
tidyr & ⭐ 必備 & pivot\_longer/wider 無可取代 \\
lubridate & ⭐ 必備 & 日期處理(已納入 tidyverse 核心) \\
glue & ⭐ 強烈推薦 & 字串插值,比 paste() 直覺 \\
stringr & ✅ 有用 & 一致的字串處理 API \\
purrr & ⚠️ 可選 & R 4.1+ 匿名函數 \texttt{\textbackslash{}(x)}
減少需求 \\
forcats & ⚠️ 可選 & 因子處理,非必要 \\
\end{longtable}

\section{非 RStudio / Antigravity
開發環境設定}\label{ux975e-rstudio-antigravity-ux958bux767cux74b0ux5883ux8a2dux5b9a}

\subsection{VS Code or Antigravity
完整設定}\label{vs-code-or-antigravity-ux5b8cux6574ux8a2dux5b9a}

\textbf{必要元件}:

\begin{enumerate}
\def\labelenumi{\arabic{enumi}.}
\tightlist
\item
  \textbf{vscode-R 擴展}(REditorSupport):VS Code 市集安裝
\item
  \textbf{languageserver}:\texttt{install.packages("languageserver")}
\item
  \textbf{radian}:\texttt{pip\ install\ radian}(增強型 R 終端機)
\item
  \textbf{httpgd}:\texttt{devtools::install\_github("nx10/httpgd")}(⚠️
  2025.04 從 CRAN 移除)
\end{enumerate}

\textbf{settings.json 設定}:

\begin{Shaded}
\begin{Highlighting}[]
\FunctionTok{\{}
    \DataTypeTok{"r.rterm.mac"}\FunctionTok{:} \StringTok{"/path/to/radian"}\FunctionTok{,}
    \DataTypeTok{"r.bracketedPaste"}\FunctionTok{:} \KeywordTok{true}\FunctionTok{,}
    \DataTypeTok{"r.plot.useHttpgd"}\FunctionTok{:} \KeywordTok{true}\FunctionTok{,}
    \DataTypeTok{"r.session.levelOfObjectDetail"}\FunctionTok{:} \StringTok{"Detailed"}
\FunctionTok{\}}
\end{Highlighting}
\end{Shaded}

\subsection{Neovim 設定(R.nvim)}\label{neovim-ux8a2dux5b9ar.nvim}

\textbf{Nvim-R 已被 R.nvim 取代},Neovim 使用者應遷移:

\begin{Shaded}
\begin{Highlighting}[]
\CommentTok{{-}{-} lazy.nvim 設定}
\FunctionTok{require}\OperatorTok{(}\StringTok{"lazy"}\OperatorTok{).}\NormalTok{setup}\OperatorTok{(\{}
    \OperatorTok{\{} \StringTok{"R{-}nvim/R.nvim"}\OperatorTok{,} \VariableTok{lazy} \OperatorTok{=} \KeywordTok{false} \OperatorTok{\},}
    \StringTok{"R{-}nvim/cmp{-}r"}\OperatorTok{,}
    \OperatorTok{\{}
        \StringTok{"nvim{-}treesitter/nvim{-}treesitter"}\OperatorTok{,}
        \VariableTok{config} \OperatorTok{=} \KeywordTok{function}\OperatorTok{()}
            \FunctionTok{require}\OperatorTok{(}\StringTok{"nvim{-}treesitter.configs"}\OperatorTok{).}\NormalTok{setup}\OperatorTok{(\{}
                \VariableTok{ensure\_installed} \OperatorTok{=} \OperatorTok{\{} \StringTok{"r"}\OperatorTok{,} \StringTok{"markdown"}\OperatorTok{,} \StringTok{"rnoweb"} \OperatorTok{\},}
                \VariableTok{highlight} \OperatorTok{=} \OperatorTok{\{} \VariableTok{enable} \OperatorTok{=} \KeywordTok{true} \OperatorTok{\},}
            \OperatorTok{\})}
        \KeywordTok{end}\OperatorTok{,}
    \OperatorTok{\},}
\OperatorTok{\})}
\end{Highlighting}
\end{Shaded}

\subsection{Positron IDE:Posit
的新選擇}\label{positron-ideposit-ux7684ux65b0ux9078ux64c7}

\textbf{Positron}(https://posit.co/products/open-source/positron/)於
2025 年 8 月釋出穩定版,基於 VS Code 但專為資料科學優化:

\begin{itemize}
\tightlist
\item
  R 和 Python 無需安裝擴展即可使用
\item
  內建資料框檢視器(類似 RStudio / Antigravity 的 View())
\item
  Posit 官方開發維護
\end{itemize}

\textbf{適合場景}:R + Python 多語言專案、想要 RStudio / Antigravity
功能但偏好 VS Code 介面者。

\section{實用工具套件:2025
必裝清單}\label{ux5be6ux7528ux5de5ux5177ux5957ux4ef62025-ux5fc5ux88ddux6e05ux55ae}

\subsection{資料清理}\label{ux8cc7ux6599ux6e05ux7406}

\begin{longtable}[]{@{}lll@{}}
\toprule\noalign{}
套件 & 功能 & 最後更新 \\
\midrule\noalign{}
\endhead
\bottomrule\noalign{}
\endlastfoot
\textbf{janitor} & clean\_names()、tabyl()、get\_dupes() & 2024.12 \\
\textbf{skimr} & skim() 超越 summary() 的資料摘要 & 活躍 \\
\textbf{tidylog} & dplyr 操作即時回饋 & 活躍 \\
\end{longtable}

\subsection{檔案操作}\label{ux6a94ux6848ux64cdux4f5c}

\begin{itemize}
\tightlist
\item
  \textbf{here}:可重現的路徑管理,\texttt{here("data",\ "clinical.csv")}
\item
  \textbf{fs}:跨平台檔案操作,取代 base R 的 file.* 函數
\end{itemize}

\subsection{報告生成:Quarto 是 2025
新標準}\label{ux5831ux544aux751fux6210quarto-ux662f-2025-ux65b0ux6a19ux6e96}

\textbf{Posit 官方立場}:R Markdown 不會棄用但新功能只加入 Quarto。

\begin{longtable}[]{@{}lll@{}}
\toprule\noalign{}
功能 & Quarto & R Markdown \\
\midrule\noalign{}
\endhead
\bottomrule\noalign{}
\endlastfoot
多語言支援 & R、Python、Julia & 以 R 為主 \\
期刊模板 & 內建多種期刊格式 & 需額外套件 \\
交叉引用 & 內建 & 需 bookdown \\
設定檔 & 單一 \_quarto.yml & 多個設定檔 \\
\end{longtable}

\textbf{建議}:新專案使用 Quarto,現有 R Markdown 專案無需急著遷移。

\subsection{臨床研究專用資源}\label{ux81e8ux5e8aux7814ux7a76ux5c08ux7528ux8cc7ux6e90}

\begin{itemize}
\tightlist
\item
  \textbf{CRAN Task View: Clinical Trials}:
  https://cran.r-project.org/view=ClinicalTrials
\item
  \textbf{Pharmaverse}(製藥業標準):admiral、rtables、teal
\item
  \textbf{R Consortium Clinical Reporting}:
  https://rconsortium.github.io/rtrs-wg/
\end{itemize}

\section{總結:2025
年臨床研究者推薦套件組合}\label{ux7e3dux7d502025-ux5e74ux81e8ux5e8aux7814ux7a76ux8005ux63a8ux85a6ux5957ux4ef6ux7d44ux5408}

\begin{Shaded}
\begin{Highlighting}[]
\CommentTok{\# 環境管理}
\FunctionTok{install.packages}\NormalTok{(}\FunctionTok{c}\NormalTok{(}\StringTok{"renv"}\NormalTok{, }\StringTok{"pak"}\NormalTok{))}

\CommentTok{\# 核心工具}
\FunctionTok{install.packages}\NormalTok{(}\FunctionTok{c}\NormalTok{(}\StringTok{"tidyverse"}\NormalTok{, }\StringTok{"here"}\NormalTok{, }\StringTok{"fs"}\NormalTok{, }\StringTok{"janitor"}\NormalTok{, }\StringTok{"skimr"}\NormalTok{, }\StringTok{"rio"}\NormalTok{))}

\CommentTok{\# 臨床表格}
\FunctionTok{install.packages}\NormalTok{(}\FunctionTok{c}\NormalTok{(}\StringTok{"gtsummary"}\NormalTok{, }\StringTok{"flextable"}\NormalTok{, }\StringTok{"gt"}\NormalTok{))}

\CommentTok{\# 視覺化}
\FunctionTok{install.packages}\NormalTok{(}\FunctionTok{c}\NormalTok{(}\StringTok{"ggpubr"}\NormalTok{, }\StringTok{"ggsci"}\NormalTok{, }\StringTok{"patchwork"}\NormalTok{, }\StringTok{"ggsurvfit"}\NormalTok{, }\StringTok{"forestploter"}\NormalTok{))}

\CommentTok{\# 流程圖}
\FunctionTok{install.packages}\NormalTok{(}\FunctionTok{c}\NormalTok{(}\StringTok{"flowchart"}\NormalTok{, }\StringTok{"consort"}\NormalTok{))}

\CommentTok{\# 報告}
\FunctionTok{install.packages}\NormalTok{(}\StringTok{"quarto"}\NormalTok{)}
\end{Highlighting}
\end{Shaded}

\subsection{已過時或不建議使用的套件}\label{ux5df2ux904eux6642ux6216ux4e0dux5efaux8b70ux4f7fux7528ux7684ux5957ux4ef6}

\begin{longtable}[]{@{}lll@{}}
\toprule\noalign{}
套件 & 狀態 & 替代方案 \\
\midrule\noalign{}
\endhead
\bottomrule\noalign{}
\endlastfoot
\textbf{packrat} & 軟棄用 & renv \\
\textbf{survminer} & 開發較不活躍 & ggsurvfit \\
\textbf{plyr} & 已被取代 & dplyr + purrr \\
\textbf{reshape2} & 已被取代 & tidyr \\
\textbf{drake} & 已被取代 & targets \\
\textbf{Nvim-R} & Neovim 用戶應遷移 & R.nvim \\
\end{longtable}




\end{document}
